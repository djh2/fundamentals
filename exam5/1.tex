% You must \input this file into a file the has previously
% defined the \ifrubric conditional.

\documentclass[12pt]{article}                   % -*- latex -*-

%%%%%%%%%%%%%%%%%%%%%%%%%%%%%%%%%%%%%%%%%%%%%%%%%%%%%%%%%%%%%%%%%%%%%%%%%%%%%%%
\usepackage{./exam}
\usepackage{type1cm} % We are using scalable fonts. Really.
\usepackage{alltt,times,comment,amsmath}
\usepackage{alltt}
\usepackage{ct} % tight CM font for \tt
\usepackage[usenames,dvipsnames]{xcolor}
\newexercise{exercise}{Exercise}[subsection]

\makeatletter
\newcommand{\ie}{\mbox{\emph{i.e.}}}    % \mbox keeps the last period from
\newcommand{\Ie}{\mbox{\emph{I.e.}}}    % looking like an end-of-sentence.
\newcommand{\etc}{\emph{etc.}}

\newcommand{\itum}[1]{\item{\bf #1}\\*}

\newcommand{\var}[1]{\textrm{\textit{#1}}}

\newenvironment{inset}
 {\bgroup\parskip=1ex plus 1ex\begin{list}{}%
        {\topsep=0pt\rightmargin\leftmargin}%
        \item[]}%
 {\end{list}\leavevmode\egroup\global\@ignoretrue}

\newenvironment{insetverb}%
  {\begin{inset}\begin{verbatim}}%
  {\end{verbatim}\end{end}}%

\def\pts#1{\marginpar{\footnotesize \raggedright  \fbox{#1 {\sc Points}}}}
%\def\pts#1{\relax}

\ifrubric
\newenvironment{solution}{\color{Red}}{}
\else
\excludecomment{solution}
\fi
\newexercise{problem}{Problem}

\newcommand\code[1]{\texttt{#1}}

% Works in math mode; all special chars remain special; cheaper than \cd.
% Will not be correct size in super and subscripts, though.
\newcommand{\ex}[1]{\mbox{\ttt #1}} 
\makeatother

%%%%%%%%%%%%%%%%%%%%%%%%%%%%%%%%%%%%%%%%%%%%%%%%%%%%%%%%%%%%%%%%%%%%%%%%%%%%%%%
\begin{document}
\vspace*{-1.5cm}
\centerline{\Large CSU2510H Exam 1 \ifrubric (SOLUTION) \fi-- Spring 2013}

\vspace{0.5cm}

\ifrubric\relax\else
\begin{center}
\begin{tabular}{l@{\qquad}l}
Name:                        & \rule{174pt}{1pt} \\[.5cm]
Student Id (last 4 digits):  & \rule{174pt}{1pt} \\[.5cm]
%Section (morning, honors or afternoon):           & \rule{174pt}{1pt} \\[.5cm]
\end{tabular}
\end{center}
\fi

\noindent\begin{minipage}{8cm}\sloppy
\begin{itemize}
\item You may use the usual primitives and expression forms of any of
  the \code{class} languages; for everything else, define it.

\item You may write {\tt {\slshape c} $\rightarrow$ {\slshape e}} as
  shorthand for
  {\tt (check-expect {\slshape c e})}.

\item To add a method to an existing class definition, you
  may write just the method and indicate the appropriate class name
  rather than re-write the entire class definition.

\item We expect data \emph{and} interface definitions.

\item If an interface is given to you, you do not need to repeat the
  contract and purpose statements in your implementations.  Likewise,
  you do not need to repeat any test cases given to you, but you
  should add tests wherever appropriate.

%% DVH: This info is already in the table.
% \item You may obtain a maximum of 55 points: 50 for the first six
% problems; and five extra-credit points for the final problem.

\item The extra credit problem is \emph{all or nothing};  no partial
  credit will be awarded.

\item Unless specifically requested, templates and super classes are
  \emph{not} required.

\item Some basic test taking advice: Before you start answering
any problems, read \emph{every} problem, so your brain can  think
about the harder problems in the background while you knock off the easy ones.
\end{itemize}

\bigskip

\emph{Good luck!}
\end{minipage}\hfil\begin{minipage}[t]{6cm}
\rule{1cm}{0pt}\begin{tabular}{|c|l|@{/}r|}
\hline
{\bf Problem} & Points & out of \\ \hline
1 & & 19\\ \hline
2 & & 10\\ \hline
Extra & & 5 \\ \hline
{\bf Total} & & 28+5 \\ \hline
%{\bf Base}  & \multicolumn{2}{l|}{{\bf 56}} \\ \hline
\end{tabular}
\end{minipage}

\vfill\thispagestyle{empty}
\newpage

\begin{problem} \pts{??}

JSON is a lightweight data-interchange format.  It is widely used on
the web and plays exactly the role of S-Expressions in universe
programs, but for JavaScript programs.  It's used by folks like Google
and there's tons of data on the web that's available in the JSON
format.  So as to not be left out of the fun, your co-op employer has
asked you to design an object representation of JSON to incorporate
into their 2 million lines of {\tt class/2} code.  To start things off
simply, here are some examples of what we'll call ``JSON, Jr.'':

\begin{verbatim}
"this is JSON"
{ "this is JSON", "Jr." }
{ {}, { "So", "is" }, "this" }
\end{verbatim}
The first example is just a string.  The second is an ``array'' of two
JSON elements, which are both strings.  The third is another array,
but of three elements: the first is an array of zero elements, the
second is an array of two strings, and the third is a string.

\begin{enumerate}
\item Design an object-based data representation for JSON, Jr.
\newpage
\item Design a method for counting the number of strings in a JSON,
  Jr.~object.
\newpage
\item Among the 2M lines of code your employer maintains are 30,000
  lines of code devoted to implementations of the {\tt
    StringPred} interface, which is:
\begin{verbatim}
;; A StringPred implements
;; - apply : String -> Boolean
\end{verbatim}
These objects represent predicates on strings.  Now that your JSON,
Jr. library is in place, they've decided to put their predicates to
use in searching a large corpus of JSON data.  To accomodate this,
design the following method for JSON, Jr.~objects:
\begin{verbatim}
;; Find the first string in this JSON Jr object satisfying
;; given predicate, or #f if there's no such string.
;; find : StringPred -> String or #f
\end{verbatim}
\end{enumerate}

\end{problem}

\newpage
\ifrubric\else
[Here is some more space for the previous problem.]
\newpage
\fi
%%%%%%%%%%%%%%%%%%%%%%%%%%%%%%%%%%%%%%%%%%%%%%%%%%%%%%%%%%%%%%%%%%%%%%%%%%%%%%%
\begin{problem} \pts{19}

% Set/Property problem

In order to manage the distribution of ID cards, the University needs
a data structure for managing sets of numbers.  Van Horn started on
the implementation, resulting in what you see below.

\begin{verbatim}
;; A Set is one of:
;; - (new empty%)
;; - (new singelton% Number)
;; - (new union% Set Set)
;; Interpretation: a set that contains all the *unique* numbers in the data
;; and implements
;; - sum : -> Number
;;   Produce the sum of all the unique elements of this set
;; - sum/acc : Set -> Number
;;   Produce the sum of all the unique elements of this set that are
;;   *not* in the given set
;; - contains : Number -> Boolean
;;   Does this set contain the given number?
;; - union : Set -> Set
;;   Combine this set with the given set

(define-class empty%
  (define (sum) 0)
  (define (sum/acc s) 0))

(define-class singleton%
  (fields n)
  (define (sum) (this . n))
  (define (sum/acc s) (cond [(s . contains (this . n)) 0]
                            [else (this . n)])))

(define-class union%
  (fields l r)
  (define (sum) (this . sum/acc (new empty%)))
  (define (sum/acc s) (+ (this . l . sum/acc s)
                         (this . r . sum/acc (s . union (this . l))))))
\end{verbatim}

\newpage

Unfortunately, as you can see, Van Horn fell asleep before finishing,
and the deadline is today.  Help him out by implementing the
\verb|union| and \verb|contains| methods for all \verb|Set|s.


\ifrubric
\begin{alltt}
\rubricpts{1 pt -- union method}
\rubricpts{3 pts -- contains method}
;; in empty%
(define (union s) s)
(define (contains n) #f)
;; also ok:
(define (union s) (new union% this s))
;; in singleton%
(define (union s) (new union% this s))
(define (contains n) (= n (this . n)))
;; in union%
(define (union s) (new union% this s))
(define (contains n) (or (this . l . contains n)
                         (this . r . contains n)))
(define e (new empty%))
\rubricpts{2 pts -- tests}
(check-expect (e . contains 5) #f)
(check-expect ((new singleton% 5) . contains 5) #t)
(check-expect ((new singleton% 6) . contains 5) #f)
(check-expect ((new singleton% 6) . union  e . contains 5) #f)
(check-expect (e . union (new singleton% 5) . contains 5) #t)
(check-expect ((new singleton% 5) . union (new singleton% 5) . contains 5) #t)
\end{alltt}
\else
\newpage
[Here is some more space for the previous problem.]
\fi
\newpage

\verb|Set|s should satisfy the following property, called the
``union-contains'' property, for any number $n$ and
any two sets $s_1$ and $s_2$:

\begin{quote}
Given two sets $s_1$ and $s_2$, if $n$ is contained in the union of $s_1$ and $s_2$, then $n$ is either
contained in $s_1$ or it is contained in $s_2$.
\end{quote}

\begin{enumerate}
\item Codify the ``union-contains'' property as a predicate.  Write a
  test case that should pass (if your implementations work right) using the
  predicate you have defined.
\end{enumerate}

\ifrubric
\begin{alltt}
\rubricpts{1 pt -- contract & purpose}
;; union-contains : Set Set Number -> Boolean
\rubricpts{3 pts -- code}
(define (union-contains s1 s2 n)
  (check-expect (s1 . union s2 . contains n)
                (or (s1 . contains n) (s2 . contains n))))
\rubricpts{1 pt -- successful test}
(union-contains e (new singleton% 17) 17)
\end{alltt}
\else
\newpage
[Here is some more space for the previous problem.]
\fi

\newpage
\noindent
The University doesn't really trust Van Horn, and they want to make
sure that \verb|Set|s satisfy another property, called the
``union-sum'' property:

\begin{quote}
Given two sets $s_1$ and $s_2$, computing the sum of $s_1$ and the sum
of $s_2$ and then adding the two results is equal to the sum of the
union of  $s_1$ and $s_2$.
\end{quote}

\noindent
The University is considering adding this to the test suite as well.

\begin{enumerate}
\setcounter{enumi}{1}

\item Codify this property as a predicate, and write a test case that
  should pass using the predicate.

\item Is the ``union-sum'' property true for all sets?  If so,
  give an explanation.  If not, write a counter-example (i.e., a
  failing test case) using your predicate.

\end{enumerate}

\begin{solution}
\begin{alltt}
\rubricpts{1 pt -- contract & purpose}
;; union-sum : Set Set -> Boolean
\rubricpts{3 pts -- code}
(define (union-sum s1 s2 n)
  (check-expect (s1 . union s2 . sum)
                (+ (s1 . sum) (s2 . sum))))

\rubricpts{1 pt -- successful test}
;; successful test
(union-sum (new singleton% 17) e)
\rubricpts{3 pts -- failing test}
;; It's false, counterexample:
(union-sum (new singleton% 17) (new singleton% 17))
\end{alltt}
\end{solution}

\newpage
\ifrubric\else
[Here is some more space for the previous problem.]
\newpage
\fi
\end{problem}

%%%%%%%%%%%%%%%%%%%%%%%%%%%%%%%%%%%%%%%%%%%%%%%%%%%%%%%%%%%%%%%%%%%%%%%%%%%%%%%
\begin{problem} \pts{15}


% JSON problem

Here are data and class definitions for representing the files and
directories of a computer:

\begin{verbatim}
;; An Elem is one of
;; - (new file% String Number)
;; - (new dir% String LoElem)
(define-class file% (fields name size))
(define-class dir% (fields name elems))

;; A LoElem is one of
;; - (new mt%)
;; - (new cons% Elem LoElem)
(define-class mt%)
(define-class cons% (fields first rest))    
\end{verbatim}


\noindent
Here is an example of a directory tree and its representation as an
\verb|Elem|:
\begin{verbatim}
#|
 User
   +-- A
   |   +-- a.txt
   |   +-- B
   |       +-- b.rkt
   +-- C
       +-- c.c
|#

(define mt (new mt%))
(define a (new file% "a.txt" 5))
(define b (new file% "b.rkt" 10))
(define c (new file% "c.c" 1000))

(define C/ (new dir% "C" (new cons% c mt)))
(define B/ (new dir% "B" (new cons% b mt)))
(define A/ (new dir% "A" (new cons% a (new cons% B/ mt))))
(define User/ (new dir% "/" (new cons% A/ (new cons% C/ mt))))
\end{verbatim}

\newpage
\noindent
Your job is to implement one of the main tasks of the common Unix
utility \verb|find|, which gives a list of all the files within an
element.  To do so, add a \verb|list-files| method of \verb|Elem|
which produces a list of files.  So for example, we expect:
\begin{verbatim}
(check-expect (User/ . list-files) 
              (new cons% a (new cons% b (new cons% c mt))))
\end{verbatim}
\emph{Hint}: it may come in handy to be able to concatenate two
\verb|LoElem| together.

\ifrubric
\else
\newpage
[Here is some more space for the previous problem.]
\fi

\begin{solution}
\begin{alltt}
\rubricpts{1pts -- interfaces, contracts, purpose}
;; An Elem implements
;; - list-files : -> LoElem
;;   List all files within this element.

\rubricpts{2pts -- interfaces, contracts, purpose}
;; A LoElem implements
;; - list-files : -> LoElem
;;   List all files within this list of elements.
;; - append : LoElem -> LoElem
;;   Append this list of elements to given list.

\rubricpts{1pt -- code}
;; in dir%
(define (list-files)
  (this . elems . list-files))

\rubricpts{1pt -- code}
;; in file%
(define (list-files)
  (new cons% this (new mt%)))

\rubricpts{2pt -- code}
;; in mt%
(define (append loe) loe)
(define (list-files) this)

\rubricpts{4pt -- code}
;; in cons%
(define (append loe)
  (new cons% (this . first) (this . rest . append loe)))
(define (list-files)
  (this . first . list-files . append (this . rest . list-files)))

\rubricpts{1pt -- sanity}

\rubricpts{3pts -- tests}
;; tests
(check-expect (mt . append (new cons% a mt)) (new cons% a mt))
(check-expect ((new cons% a mt) . append (new cons% b mt))
              (new cons% a (new cons% b mt)))

(check-expect (mt . list-files) mt)
(check-expect ((new cons% a (new cons% b mt)) . list-files)
              (new cons% a (new cons% b mt)))

(check-expect (a . list-files) (new cons% a mt))
\end{alltt}
\end{solution}

\end{problem}
\newpage


%%%%%%%%%%%%%%%%%%%%%%%%%%%%%%%%%%%%%%%%%%%%%%%%%%%%%%%%%%%%%%%%%%%%%%%%%%%%%%%

\begin{problem}\pts{5}

% Visitor on Binary Trees

\textbf{Extra Credit}
\\
\noindent
After Van Horn discovered the problems with numbers, he attempted to
fix them.  Unfortunately, not only did he not fix them, he broke
booleans as well.  Fortunately, we have an idea about how you can
help, expressed in the following interface definitions.  

\begin{verbatim}
;; A Bool implements:
;; - branch : [Action X] [Action X] -> X
;;   run the first action if this bool is true, 
;;   otherwise run the second action

;; An [Action X] implements:
;; - run : -> X
;; run this action


\end{verbatim}

Following these interface definitions, design data and class
defintions for implementations of the \verb|Bool| interface, and
reimplement the \verb|fact| method to use them.
You will need to provide new defintions for \verb|true| and
\verb|false|, since they were broken by this catastrophe as well.

\ifrubric\else
\newpage
[Here is some more space for the previous problem.]
\fi


\begin{solution}
\begin{alltt}
\rubricpts{5 pts -- get it right}
(define-class fact%
  (define (fact n)
    (n . zero? . branch (new const% one) (new recur% n))))

(define-class const%
  (fields run))

(define-class recur%
  (fields n)
  (define (run) 
    ((new fact%) . fact (this . n . sub1) . times (this . n))))

(define-class t%
  (define (branch t e) (t . run)))
(define-class f%
  (define (branch t e) (e . run)))

(define true (new t%)) 
(define false (new f%))

(check-expect ((new fact%) . fact zero) one)                  ; 0! = 1
(check-expect ((new fact%) . fact one) one)                   ; 1! = 1
(check-expect ((new fact%) . fact (one . add1)) (one . add1)) ; 2! = 2
(check-expect ((new fact%) . fact (one . add1 . add1))        ; 3! = 6
              (one . add1 . add1 . add1 . add1 . add1))
\end{alltt}
\end{solution}
\end{problem}
\end{document}


