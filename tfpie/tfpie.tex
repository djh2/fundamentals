\documentclass[submission,copyright]{eptcs}
\providecommand{\event}{TFPIE} % Name of the event you are submitting to
\usepackage{breakurl}             % Not needed if you use pdflatex only.

\title{How to Design Programs with Class}
\author{Sam Tobin-Hochstadt \quad\qquad David Van Horn
\institute{Northeastern University\\
Boston, Massachusetts, USA}
\email{\{samth,dvanhorn\}@ccs.neu.edu}
}
\def\titlerunning{A Longtitled Paper}
\def\authorrunning{S. Tobin-Hochstadt \& D. Van Horn}
\begin{document}
\maketitle

\begin{abstract}
We propose a bridge between functional and object-oriented programming
in the first-year cirriculum.
\end{abstract}

\section{Introduction}

\emph{Thesis: ...}

\section{Background: the first semester}

\section{A small shift of focus}

The semester starts with the minimal conceptual change from the
previous semester: the addition of objects (same IDE, same syntax,
same semantics [just a new kind of value], same interactive video
games, etc.).

\subsection{Side-by-side example of program design}

[First semester vs second semester example programs.]

\section{Our course}

\subsection{Distributed, Interactive Programs}

\subsection{Representation-indepence and interface-oriented programming}

\subsection{Unifying functions and objects}

\section{From principles to fashions}

Could work with any fashion, and should really lead into many
fashionable languages.  The purpose of showing many is to demonstrate
that the same principles apply in each setting.  Each language can also
expose students to various mechanisms.

\subsection{Java}

\begin{itemize}
\item Types
\item Compiler
\item ``Industrial'' features (collections library, ...)
\end{itemize}


\section{Comparison with HtDC}

The noble goals of ProfJ:

\begin{itemize}
\item language levels
\item continuity in IDE from first semester
\end{itemize}

Many of the shortcomings of HtDC and ProfJ are ``day 1'' problems.
The transition is too abrupt and too large.

The failures of  ProfJ:

\begin{itemize}
\item changes languages from first semester rather than adding new
  concept (causes students confusing deep issue (O.O. design) and
  shallow ones (Java-like syntax) and orthogonal issues (static type
  discipline).  (This also ultimately lead to the tool being abandoned
  as it was too difficult to maintain.)

\item commits to a particular fashion rather than enabling any.

\item abandons the first semester approach and replaces it with a
  (completely) object-oriented approach on day one.  There are
  analogies between functional and object concepts (``a method is like
  a function''), but the two never co-exist in a program.

\item Java, even pared down pedagoical Java, is syntactically heavy; discourages
  interactive programming

\end{itemize}

The changes that occur on Day 1 of HtDC w/ ProfJ:

\begin{itemize}
\item Radically different syntax
\item Types
\item Total committment to objects
\item Subtly different base values
\end{itemize}

The changes that occur on Day 1 of HtDC w/o ProfJ:

\begin{itemize}
\item Different IDE
\item Compilation
\item Loss of interactivity
\end{itemize}

Some of these are premature, and better introduced later in the
course.  Some are just irrelevant trivia (different syntax) that often
favors ``experienced'' students while needlessly intimidating students
with only the previous semester under their belt.  Many of these
trivial matters, e.g. Eclipse configuration, javac command line
arguments, the CLASSPATH, can be very frustrating to the uninitiated,
causing students to believe struggling with such crap is \emph{the
  essence of computer science}.  Bright, promising students 

\section{Experience and outlook}

Perspective into the upper-class cirriculum:

\begin{itemize}
\item types, contract, invariants : formal methods, program verification,
  
\item representation independence : component-based software engineering,
  data-structures, algorithms

\item interfaces, interactions, distribution : operating systems, networking
\end{itemize}

\nocite{*}
\bibliographystyle{eptcs}
\bibliography{bibliography}
\end{document}
