\documentclass{article}

\usepackage{slatex}

\pagestyle{empty}
\setlength{\parindent}{0mm}

\begin{document}
\section*{Lab Quiz 4}

\bigskip
\bigskip
Name: \underline{\hspace*{4in}}

\bigskip
\setlength{\parskip}{8pt}

\begin{schemedisplay}
  (define TICK-RATE 1/10)

  ; A Projectile is a (new projectile\% Complex Complex)
  (define-class projectile%
    (fields location velocity))

  ; A Rock is a (new rock\% Complex Complex)
  (define-class rock%
    (super projectile%)

    ; \scheme{-> Rock}
    (define/public (tick)
      (new rock%
           (+ (field location) (* TICK-RATE (field velocity)))
           (field velocity))))

  ; A Banana is a (new banana\% Complex Complex)
  (define-class banana%
    (super projectile%)

    ; \scheme{-> Banana}
    (define/public (tick)
      (new banana%
           (+ (field location) (* TICK-RATE (field velocity)))
           (field velocity))))
\end{schemedisplay}

The two \scheme{tick} methods do basically the same thing. Abstract
\scheme{tick} up to the superclass by introducing appropriate constructor
methods in \scheme{rock%} and \scheme{banana%}.

\end{document}
