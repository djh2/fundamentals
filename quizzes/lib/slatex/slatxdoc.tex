%slatxdoc.tex
%SLaTeX Version 2.4
%Documentation for SLaTeX
%(c) Dorai Sitaram
%dorai@ccs.neu.edu

\ifx\shipout\undefined\else
\input slatex.sty
\fi

\slatexdisable{enableslatex}
\input tex2page
\input btxmac
\magnification\magstephalf
\input defun


%\advance\hoffset -.1true in
%\advance\voffset -.1 true in
%\advance\hsize .2 true in
%\advance\vsize .2 true in

\texonly
%\input 8pt
%\input 2col

\sidemargin 1.75 true in

%\input defun
\endtexonly

\htmlonly
\scmkeyword{define-macro! extend-syntax}
\endhtmlonly


%%%% other definitions

%\frenchspacing

% color, if possible

\InputIfFileExists{colordvi}{}{\let\Red\relax
\let\Blue\relax
\let\Green\relax}

%\def\section{\bigbreak\begingroup\noindent\bf
%\def\par{\endgroup\nobreak\smallskip}}

\def\bs{{\tt\char`\\}}
\def\{{\ifmmode\lbrace\else$\lbrace$\fi}
\def\}{\ifmmode\rbrace\else$\rbrace$\fi}

% sans-serif

\font\sf cmss10

% \newcheapcount\bibitemnumber

% \def\bibitem{\par\globaladvancecheapcount\bibitemnumber 1%
% \edef\recentlabel{\bibitemnumber}%
% [\bibitemnumber]\label}

%
\overfullrule 0pt

%\parindent1.5em
\pretolerance0
\tolerance4000

%\raggedbottom

\hyphenation{def-sla-tex-env-style dorai env-name
kyo-to left-code-skip right-code-skip
scheme-case-sen-si-tive scheme-dis-play scheme-in-put
scheme-re-gion scheme-re-sponse-box scheme-re-sult SLaTeX}


\texonly
\def\re{\defun{}} % glossary items
%\hyphenchar\eighttt=-1
\endtexonly

\def\wideline{\par\centerline{\hrulefill}\par}

\let\footnotehook\eightpoint

% abbrevs

\let\n\noindent
\let\f\numberedfootnote

%\packindex

\leftcodeskip\parindent

\enableslatex

\title{How to Use SLaTeX}

%\bigskip

\centerline{\urlh{http://www.ccs.neu.edu/~dorai}{Dorai Sitaram}}
\centerline{\urlh{slatex.tar.bz2}{\htmlonly Download
\endhtmlonly Version \input ./version }}
%\centerline{{\tt dorai@cs.rice.edu}}
%\centerline{Department of Computer Science}
%\centerline{Rice University}
%\centerline{Houston, TX 77251-1892}

\bigskip

\n SLaTeX \index{introduction} is a Scheme program that allows
you to write programs or program fragments ``as is'' in your
TeX or LaTeX source.  It is particularly geared to the
programming languages Scheme and other Lisps, eg, Common
Lisp.  The formatting of the code includes assigning
appropriate fonts and colors to the various tokens in the
code (keywords, variables, constants, data), at the same
time retaining the proper indentation when going to the
non-monospace (non-typewriter) fonts provided by TeX.
SLaTeX comes with two databases that recognize the
identifier conventions of Scheme and Common Lisp
respectively. These can be modified by the user with easy
TeX commands.  In addition, the user can tell SLaTeX to
typeset certain identifiers as specially suited TeX
expressions (ie, beyond just fonting them).  All this is
done without interfering with the identifier conventions of
the language of the programming code.  In sum, no change
need be made to your (presumably running) program code in
order to get a typeset version with a desired look: You can
get a spectrum of styles ranging from {\it no\/} fonting
through basic default fonting to various
``math\-e\-mat\-i\-cal''-looking output for pedagogic or
other reasons.

Popular packages for typesetting Scheme code, eg,
Queinnec's LiSP2TeX~\cite{lisp2tex}
and Ramsdell's SchemeWeb~\cite{schemeweb}, use
a {\tt verbatim}-like environment
where all the characters are in a {\tt monospace typewriter}
font.  While the {\tt monospace} ensures that the indentation
is not affected, it fails to distinguish between the various
tokens used in the code.  On the other hand, SLaTeX can
font- or even color-code\f{To get color, you need a dvi
driver such as  Rokicki's {\tt dvips} and a color printer.  If
you don't have a color printer, you can still view your
color-coded document on screen, but will have to settle for
grayscaling on your paper output.} the tokens, eg, {\bf
boldface} and/or red for keywords like \scheme{define} and
\scheme{lambda}, {\sf sans-serif} and/or green for
constants like \scheme{#t} and \scheme{42}, and {\it italics\/}
and/or blue for variables such as
\scheme{x} and \scheme{y} in \scheme{(lambda (x y)
(cons x (cons y '())))}.  SLaTeX provides a
convenient way of capturing the indentation
information as well as assigning distinguishing fonts
to code tokens without requiring the user to worry
about fonting and spacing.  It uses temporary files
to store its typeset version of the user's code
fragments and then calls TeX or LaTeX on the user's
TeX files as well as these temporaries.

\slatexdisable{enableslatex}

The rest of this manual has the following outline:
%section~\ref{intro-example} introduces a small example
%that covers the SLaTeX control sequences you will need
%the most.  Both LaTeX~\cite{latex} and plain
%TeX~\cite{tex} versions are
%shown. Section~\ref{slatex-sty-files} describes the
%SLaTeX style files {\tt slatex.sty} and {\tt cltl.sty}.
%Section~\ref{glossary} provides a complete description
%of all the SLaTeX control sequences.  These include
%commands for manipulating output positioning, enhancing
%the database, changing the fonting defaults, adding
%special symbols, and selective disabling of
%SLaTeX. Section~\ref{preamble} offers guidelines for
%setting up a preamble that reflects your typesetting
%taste. Section~\ref{download} contains information on
%obtaining and installing SLaTeX.

\tableofcontents

\beginchapter 1 Introductory example

\label{intro-example}
\index{quick introduction}
We now present a short text that illustrates the
most basic features of SLaTeX.  Luckily, these
cover the commands you will need and use the most.
There are a couple of minor differences between the
ways SLaTeX is used in LaTeX and in plain TeX (but
see the entry in section~\ref{glossary} for \p{\defslatexenvstyle} if
you wish to use the plain-TeX style with the LaTeX
format, or the LaTeX style with the plain format).

\beginsection {SLaTeX for LaTeX users}

\index{LaTeX}
\index{scheme@{\tt\bs scheme}}
\index{schemedisplay@{\tt\{schemedisplay\}}!in LaTeX}
\index{in-text Scheme code}
\index{displayed Scheme code}
\index{slatex.sty@{\tt slatex.sty}}
\index{slatex.sty@{\tt slatex.sty}!as document style}
Consider the following LaTeX ({\it and\/}
SLaTeX) file
{\tt quick.tex}:

\goodbreak

---

\verb+
% quick.tex
\documentclass{article}
\usepackage{slatex}
%
%That was LaTeX2e.  If you use
%LaTeX2.09, or LaTeX2e in
%2.09 compatibility mode, use
%
%\documentstyle[slatex]{article}
%
%or
%
%\documentstyle{article}
%\input slatex.sty
\begin{document}

In Scheme, the expression
\scheme|(set! x 42)| returns
an unspecified value, rather
than \scheme'42'.  However,
one could get a \scheme{set!}
of the latter style with:

\begin{schemedisplay}
(define-syntax setq
  (syntax-rules ()
    [(setq x a)
     (begin (set! x a)
            x)]))
\end{schemedisplay}

\end{document}
+

---

First, the SLaTeX definitions in the style file
{\tt slatex.sty} are loaded into your LaTeX file. This may
be done either via \p{\usepackage} or
\p{\input} --- or, if you use LaTeX 2.09, as a
\p{\documentstyle} option.

\index{scheme@{\tt\bs scheme}!using grouped argument}

In-text code is introduced by the SLaTeX control
sequence \p{\scheme} and is flanked by a pair of
identical characters that are not alphabetic letters
or ``\p|{|''.  As a special convenient case,
SLaTeX also allows the form \p{\scheme{...}}.

The SLaTeX control sequences for
displayed code are the opening

\verb{
\begin{schemedisplay}
}

\n and the closing

\verb{
\end{schemedisplay}
}

The file is now SLaTeX'd by running the command {\tt slatex}
on it from the Unix or DOS or OS/2 command
line:

\verb{
slatex quick
}

\n or

\verb{
slatex quick.tex
}

\n Alternatively, you can load the file {\tt callsla.scm} into
your Scheme or Common Lisp, and then call

\verb{
(call-slatex "quick")
}

\n This calls a Scheme program {\tt slatex.scm} that
typesets the Scheme code fragments from {\tt quick.tex} into
temporary files.  Thereafter, {\tt quick.tex} along with the
temporary files are passed
to LaTeX.  (For information on judiciously
reusing temporary files, see
\p{\separateincludes}.)
The resulting
{\tt quick.dvi} file, when viewed or printed, looks like:

\enableslatex

---

\n
In Scheme, the expression
\scheme|(set! x 42)| returns
an unspecified value, rather
than \scheme'42'.  However,
one could get a \scheme{set!}
of the latter style with:

\schemedisplay
(define-syntax setq
  (syntax-rules ()
    [(setq x a)
     (begin (set! x a)
            x)]))
\endschemedisplay

---

\index{recognizing new syntactic keywords automatically}

Note that \scheme{setq}, although not normally a
syntactic keyword in Scheme, is nevertheless
automatically recognized as such because of the context
in which it occurs.  No special treatment is needed to
ensure that it will continue be treated as such in any
subsequent Scheme code in the document.
\slatexdisable{enableslatex}

\beginsection {SLaTeX for plain TeX users}

\index{plain TeX}
\index{scheme@{\tt\bs scheme}}
\index{schemedisplay@{\tt\{schemedisplay\}}!in plain TeX}
\index{in-text Scheme code}
\index{displayed Scheme code}
SLaTeX works much the same way with plain TeX as with
LaTeX, but for only two exceptions.  First, since plain TeX
doesn't have \p{\documentstyle}, the file
{\tt slatex.sty} must be introduced via an \p{\input}
statement before its commands can be used in the plain
TeX source.

\index{environments}

Second, since plain TeX does not have LaTeX's
\p|\begin{|{\it env\/}\p|} ... \end{|{\it env\/}\p|}|
style of environments, any
environment commands in SLaTeX are invoked with the
opening \p{\}{\it env\/} and the closing
\p{\end}{\it env}.

The plain TeX version of {\tt quick.tex} looks like:

---

\verb+
% quick.tex
\input slatex.sty

In Scheme, the expression
\scheme|(set! x 42)| returns
an unspecified value, rather
than \scheme'42'.  However,
one could get a \scheme{set!}
of the latter style with:

\schemedisplay
(define-syntax setq
  (syntax-rules ()
    [(setq x a)
     (begin (set! x a)
            x)]))
\endschemedisplay
\bye
+

---

The file is now SLaTeX'd by invoking {\tt slatex} as
before --- SLaTeX is clever enough to figure out
whether the file it operates on should later be sent to
LaTeX or plain TeX.

\beginchapter 2 The SLaTeX style files

\label{slatex-sty-files}
\index{slatex.sty@{\tt slatex.sty}}
The LaTeX (or TeX) file that is given to SLaTeX
undergoes some code-setting preprocessing and is then
handed over to LaTeX (or TeX).  The style file
{\tt slatex.sty} defines the appropriate commands so that
LaTeX (or TeX) can recognize the SLaTeX-specific
directives and deal with them appropriately.  As
mentioned above, you may
either \p{\input} the file {\tt slatex.sty},
or use it as the \p{\documentstyle} option
{\tt slatex}.

\index{cltl.sty@{\tt cltl.sty}}
\index{SLaTeX database!for Scheme}
\index{SLaTeX database!for Common Lisp}
\index{SLaTeX database!modifying}

The default database of SLaTeX recognizes the keywords
and constants of Scheme.  The database can be modified
with the commands \p{\setkeyword},
\p{\setconstant}, \p{\setvariable},
\p{\setdata},
\p{\setspecialsymbol} and \p{\unsetspecialsymbol}
(qv).  If you're using Common Lisp rather than
Scheme, use {\tt cltl.sty} instead of {\tt slatex.sty}.
{\tt cltl.sty} loads {\tt slatex.sty} and modifies the
database to reflect Common Lisp.  You may also fashion your
own {\tt .sty} files on the model of {\tt cltl.sty}.

\beginchapter 3 SLaTeX's control sequences

\label{glossary}
\index{SLaTeX control sequences}
You've already seen the SLaTeX control sequence
\p{\scheme} and the environment
\p{{schemedisplay}}.\f{The notation
\p|{|{\it envname\/}\p|}|
is used to denote an environment
named {\it envname}.  Usually, in plain TeX, this is the pair
\p{\}{\it envname\/} and \p{\end}{\it envname}, while in
LaTeX, it is the environment formed with
\p|\begin{|{\it envname\/}\p|}| and
\p|\end{|{\it envname\/}\p|}|.
But see
\p{\defslatexenvstyle}.} These suffice for quite a
few instances of handling code.  However, you will
occasionally require more control on the typesetting
process.  To help you in such situations, here is a
complete
list of SLaTeX control sequences with
examples.  These include
commands for manipulating output positioning, enhancing
the database, changing the fonting defaults, adding
special symbols, and selective disabling of
SLaTeX.

\index{schemedisplay@{\tt\{schemedisplay\}}}
\index{displayed Scheme code}
\re{\p{{schemedisplay}}}

Typesets the enclosed code, which is typically several
lines of code indented as is common in Scheme files.  Eg,

%\evalh{(set! *use-img-for-in-text-math?* #f)}

\verb{
\begin{schemedisplay}
(define compose
  ;this is also known as $B$
  (lambda (f g)
    (lambda (x)
      (apply f (g x)))))
\end{schemedisplay}
is the ``compose'' function.
}

\n produces

\enableslatex
\schemedisplay
(define compose
  ;this is also known as $B$
  (lambda (f g)
    (lambda (x)
      (apply f (g x)))))
\endschemedisplay
is the ``compose'' function.
\slatexdisable{enableslatex}

As with all LaTeX environment enders, if the line
after \p{\end{schemedisplay}} contains
non-whitespace text, the paragraph continues.
Otherwise --- ie, when \p{\end{schemedisplay}}
is followed by at least one blank line --- a fresh
paragraph is started.  Similarly, in plain TeX, a
fresh paragraph is started after a
\p{{schemedisplay}} only if
\p{\endschemedisplay} is followed by at least one
blank line.

\index{Scheme comments}

Comments in Scheme are usually introduced by ``{\tt ;}''
(semicolon).  The rest of the line after a ``{\tt ;}''
is set in paragraph mode.

\index{TeX paragraphs amidst Scheme code}

Separate {\it blocks\/} of code can either be
introduced in different \p{{schemedisplay}}
environments or put in a single
\p{{schemedisplay}} and separated by a line with a
``{\tt ;}'' in the first column.  This ``{\tt ;}'' is
not typeset and anything following it on the line is
set in paragraph mode.  Consecutive lines with ``{\tt ;}'' in
the first column are treated as input for a
TeX paragraph, with words possibly moved around from
line to line to ensure justification. When in
paragraph mode, the first line that has {\it no\/}
leading ``{\tt ;}'' signals a fresh block of Scheme
code within the \p{{schemedisplay}}.  (The
\p{{schemedisplay}} may end, or commence, on
either a paragraph or a Scheme code block.)

Eg,

\verb{
\begin{schemedisplay}
(define even?
  ;testing evenness
  (lambda (n)
    (if (= n 0) #t
        (not (odd? (- n 1))))
  ))
; The procedures {\it even?\/}
; above and {\it odd?\/} below
; are mutually recursive.
(define odd?
  ;testing oddness
  (lambda (n)
    (if (= n 0) #f
        (not (even? (- n 1))))
  ))
\end{schemedisplay}
}

\n produces

\enableslatex
\schemedisplay
(define even?
  ;testing evenness
  (lambda (n)
    (if (= n 0) #t
        (not (odd? (- n 1))))
  ))
; The procedures {\it even?\/}
; above and {\it odd?\/} below
; are mutually recursive.
(define odd?
  ;testing oddness
  (lambda (n)
    (if (= n 0) #f
        (not (even? (- n 1))))
  ))
\endschemedisplay

\slatexdisable{enableslatex}
SLaTeX can recognize that blocks of code are separate
if you have at least one empty line separating them.
Ie, there is no need for empty ``{\tt ;}'' lines.
This convenience is to accommodate Scheme files where
definitions are usually separated by one or more
blank lines.

\index{schemedisplay@{\tt\{schemedisplay\}}!allowing page
breaks in}

Intervening paragraphs, either with lines with a
leading ``{\tt ;}'', or with blank lines, are ideal
spots for \p{{schemedisplay}} to allow pagebreaks.
In fact, the default setting for
\p{{schemedisplay}} also allows pagebreaks {\it within\/} a
Scheme block, but it is easy to disable
this (see entry for
\p{\rightcodeskip}).

{\tolerance100000 The space surrounding displayed
Scheme code can be modified by setting the {\it skips\/}
\p{\abovecodeskip}, \p{\belowcodeskip}, \p{\leftcodeskip},
and \p{\rightcodeskip} (qv).\par}

Note: see \p{{schemeregion}}.

\index{schemeresponse@{\tt\{schemeresponse\}}}
\index{displayed Scheme result}
\re{\p{{schemeresponse}}}

This is like \p{{schemedisplay}}, except that the
code is displayed as the output of a Scheme
evaluation, ie, as data.  In other words, keyword
and variable fonts are disabled.

Note: see \p{\schemeresult} and
\p{{schemeresponsebox}}.

\index{scheme@{\tt\bs scheme}}
\index{in-text Scheme code}
\re{\p{\scheme}}

{\tolerance100000 Typesets its argument, which is
enclosed in arbitrary but identical non-alphabetic and
non-\p|{| characters, as in-text code.  Special case:
\p{\scheme{...}} is a convenience (provided the `\p{...}'\
doesn't contain a `\p|}|').  Eg, \p+\scheme|(call/cc+
\p+(lambda (x) x))|+ and \p+\scheme{(call/cc+ \p+(lambda (x)
x))}+ both produce
\enableslatex
\scheme{(call/cc (lambda (x) x))}.
\slatexdisable{enableslatex}
\index{scheme@{\tt\bs scheme}!using grouped argument}
\par}

\index{nesting SLaTeX control sequences}
It {\it is\/} permitted to intermix calls to
\p{{schemedisplay}} and
\p{\scheme}.  Thus,

\verb{
\begin{schemedisplay}
(define factorial
  (lambda (n)
    (if (= n 0)
        ;\scheme{(zero? n)}
        ;also possible
        1 (* n (factorial
                 (- n 1))))))
    ;or \scheme{... (sub1 1)}
\end{schemedisplay}
}

\n produces

\enableslatex
\schemedisplay
(define factorial
  (lambda (n)
    (if (= n 0)
        ;\scheme{(zero? n)}
        ;also possible
        1 (* n (factorial
                 (- n 1))))))
   ;or \scheme{... (sub1 1)}
\endschemedisplay

\slatexdisable{enableslatex}
Note: see \p{{schemeregion}}.

\index{schemeresult@{\tt\bs schemeresult}}
\re{\p{\schemeresult}}

Typesets its argument, which is enclosed in arbitrary
but identical non-alphabetic and non-\p|{|
characters, as in-text Scheme ``result'' or data:
ie, keyword and variable fonts are disabled.
Special convenient case (as for \p{\scheme}):
\p{\schemeresult{...}}.  Eg,
\index{schemeresult@{\tt\bs schemeresult}!using grouped argument}

\verb+
\scheme|((lambda () (cons 'lambda
'cons)))| yields
\schemeresult|(lambda . cons)|.
+

\n produces

\enableslatex
\scheme|((lambda () (cons 'lambda
'cons)))| yields
\schemeresult|(lambda . cons)|.
\slatexdisable{enableslatex}

\index{schemebox@{\tt\{schemebox\}}}
\index{boxed Scheme code}
\re{\p{{schemebox}}}

The \p{{schemebox}} environment is similar to
\p{{schemedisplay}} except that the code is
provided as a ``box'' (ie, it is not ``displayed''
in the standard way).  Indeed, when the appropriate
skip parameters are set, \p{{schemedisplay}}
itself {\it may\/}\f{Yes, {\it may\/}: Not all
\p{{schemedisplay}}s invoke \p{{schemebox}},
and if you're curious why, see entry for
\p{\rightcodeskip}.  It is a matter of whether
pagebreaks within Scheme code are allowed or not.}
use a \p{{schemebox}} to create a box of code that
is set off with all-round space as a display.

Saving a \p{{schemebox}} in an explicit box allows you
to move your typeset code arbitrarily.

Note: see \p{{schemeregion}}.

\index{schemeresponsebox@{\tt\{schemeresponsebox\}}}
\re{\p{{schemeresponsebox}}}

This is like \p{{schemebox}}, except that the
contents are displayed as Scheme data.  See also
\p{\schemeresult} and \p{{schemeresponse}}.

\index{schemeinput@{\tt\bs schemeinput}}
\index{inputting Scheme files as is}
\re{\p{\schemeinput}}

This can be used to input Scheme files as typeset code.
(Unlike (La)TeX's \p{\input}, \p{\schemeinput}'s
argument must always be grouped.)  The Scheme file can
be specified either by its full name, or without its
extension, if the latter is \p{.scm}, \p{.ss} or
\p{.s}.  Eg,

\verb{
\schemeinput{evenodd.scm}
% the .scm is optional!
}

\n {\tolerance100000 (where \p{evenodd.scm} is the
name of a Scheme file containing the code for
\enableslatex
\scheme{even?} and \scheme{odd?} above) produces the same
effect as the
\slatexdisable{enableslatex}
\p{{schemedisplay}} version.\par}

Note: see \p{{schemeregion}}.

\index{schemeregion@{\tt\{schemeregion\}}}
\index{nesting SLaTeX control sequences}
\re{\p{{schemeregion}}}

{\tolerance100000
Calls to \p{\scheme}, \p{\schemeresult},
\p{{schemedisplay}}, \p{{schemebox}} or
\p{\schemeinput} can be nested in (a Scheme comment)
of other calls.  In (La)TeX source, they can occur in
bodies of environments or be otherwise grouped.
However, they cannot normally be passed as arguments to
macros or included in bodies of macro definitions (even
though these are complete calls and not parameterized
with respect to macro arguments).  To be able to do
these things, you should first cordon off such a text with
the
\p{{schemeregion}} environment.  SLaTeX is fairly
generous about where exactly you throw the cordon.\par}

Eg, you cannot have

\verb{
...
The code fragment $\underline
{\hbox{\scheme{(call/cc I)}}}$ is ...
...
}

\n but you {\it can\/} have

\verb{
\begin{schemeregion}
...
The code fragment $\underline
{\hbox{\scheme{(call/cc I)}}}$ is ...
...
\end{schemeregion}
}

\n and this will produce

\enableslatex
\schemeregion
\dots

The code fragment $\underline
{\hbox{\scheme{(call/cc I)}}}$ is \dots

\dots
\endschemeregion
\slatexdisable{enableslatex}

Thus, the \p{{schemeregion}} environment makes it
possible to put SLaTeX-specific commands inside macro
arguments or macro definitions without causing
rupture. Normally, this can't be done since once
SLaTeX is done preprocessing your text, all
SLaTeX-specific commands correspond to \p{comment}-like
regions --- the actual typeset code is
in an external, temporary file.  These \p{comment}
regions share the characteristic of LaTeX's {\tt verbatim}
regions, which also can't appear in macro
arguments or definitions.

To solve this, you enclose the offending text in a
\p{{schemeregion}} environment.  This ``inlines''
all the the enclosed calls to SLaTeX as actual
typeset code instead of treating such calls as \p{comment}
regions --- thus escaping the fate described
above.  A \p{{schemeregion}} is a perfect no-op as
far as the enclosed {\it non\/}-SLaTeX commands are
concerned.

However, while a \p{{schemeregion}} allows its
enclosed SLaTeX commands to be included in macro
arguments and bodies, it itself cannot be so
included.  This is because \p{{schemeregion}} is
now a super-\p{comment}. The solution is simple:
just pull the \p{{schemeregion}} cordon as outward
as possible so that all the SLaTeX calls that you
need treated specially are enclosed.  A safe approach
is to call \p{{schemeregion}} at the ``top
level'', ie, outside any nesting of groups --- it
does not hurt that the cordon is too wide.  Indeed,
you may even wrap each of your TeX files in one huge
\p{{schemeregion}} if you so wish. This will cover
any obscure ``non-robust''\f{The term ``robust'' is not
necessarily used in the same sense as in the LaTeX manual
\cite{latex}.} use of the SLaTeX primitives --- the only
downside is that SLaTeX may run slower.

Note: TeX files that are loaded using \p{\input}
from within a \p{{schemeregion}} will not
automatically inherit the robust lifestyle.  A SLaTeX
command is made robust only by an enclosing
\p{{schemeregion}} {\it in the same file as
itself}.  In other words, region markers have textual
or ``lexical'' scope, not ``dynamic'' scope.

\index{setkeyword@{\tt\bs setkeyword}}
\index{setconstant@{\tt\bs setconstant}}
\index{setvariable@{\tt\bs setvariable}}
\index{setdata@{\tt\bs setdata}}
\index{SLaTeX database!modifying}
\re{\p{\setkeyword}
\p{\setconstant}
\p{\setvariable}
\p{\setdata}}

SLaTeX has a database containing information about
which code tokens are to be treated as {\bf keywords},
which as {\sf constants}, which as {\it variables}, and
which as {\sf data}.  However, it is very likely that
you will want to add your own tokens to these
categories.  The control sequences that enable you to
do this are \p{\setkeyword},
\p{\setconstant},
\p{\setvariable}, and \p{\setdata}.  Their arguments are entered
as a (space-separated) list enclosed in braces
(\p{{}}): SLaTeX learns that these are henceforth to
be typeset in the appropriate category.  Eg,

\enableslatex
\verb{
\setconstant{infinity -infinity}
}

\n tells SLaTeX that \scheme{infinity} and
\scheme{-infinity} are to be typeset as constants.
\slatexdisable{enableslatex}

\index{recognizing new syntactic keywords automatically}

The user need not use \p{\setkeyword} to specify such
new keywords as are introduced by Scheme's and Common
Lisp's syntactic definition facilities
\enableslatex
(\scheme{define-syntax}/\scheme{syntax-rules},
\scheme{defmacro}, \scheme{extend-syntax},
\scheme{define-macro!}.) SLaTeX automatically recognizes
new macros and auxiliary keywords defined using these
facilities.
\slatexdisable{enableslatex}

In addition, quoted material is recognized as
``constant'', and strings, numbers, booleans and
characters are recognized as ``data'' without the need
to identify them with \p{\setconstant} and
\p{\setdata} respectively.

%\evalh{(set! *use-img-for-in-text-math?* #t)}
\index{setspecialsymbol@{\tt\bs setspecialsymbol}}
\index{unsetspecialsymbol@{\tt\bs unsetspecialsymbol}}
\index{SLaTeX database!modifying}
\index{recognizing special symbols}
\re{\p{\setspecialsymbol}
\p{\unsetspecialsymbol}}

These commands are useful to generate
``math\-e\-mat\-i\-cal''-looking typeset versions of
your code, over and beyond the fonting capabilities
provided by default.  Although your program code is
naturally restricted to using ascii identifiers that
follow some convention, the corresponding typeset
code could be more mnemonic and utilize the full
suite of mathematical and other symbols provided by
TeX. This of course should not require you to
interfere with your code itself, which should run in
its ascii representation.  It is only the typeset
version that has the new look.  For instance, you
might want all occurrences of \p{lambda}, \p{and},
\p{equiv?}, \p{below?}, \p{above?}, \p{a1} and \p{a2} in
your code to be typeset as
$\lambda$, $\land$, $\equiv$, $\subseteq$,
$\supseteq$, $a_1$ and $a_2$ respectively.  To do
this, you should \p{\setspecialsymbol} the
concerned identifier to the desired TeX expansion,
viz.,

\verb{
\setspecialsymbol{lambda}{$\lambda$}
\setspecialsymbol{and}{$\land$}
\setspecialsymbol{equiv?}{$\equiv$}
\setspecialsymbol{below?}{$\subseteq$}
\setspecialsymbol{above?}{$\supseteq$}
\setspecialsymbol{a1}{$a_1$}
\setspecialsymbol{a2}{$a_2$}
}
\enableslatex
\setspecialsymbol{lambda}{$\lambda$}
\setspecialsymbol{and}{$\land$}
\setspecialsymbol{equiv?}{$\equiv$}
\setspecialsymbol{below?}{$\subseteq$}
\setspecialsymbol{above?}{$\supseteq$}
\setspecialsymbol{a1}{$a_1$}
\setspecialsymbol{a2}{$a_2$}

\n Now, typing\slatexdisable{enableslatex}

\verb{
\begin{schemedisplay}
(define equiv?
  (lambda (a1 a2)
    (and (below? a1 a2)
         (above? a1 a2))))
\end{schemedisplay}
}

\n produces

\enableslatex
\schemedisplay
(define equiv?
  (lambda (a1 a2)
    (and (below? a1 a2)
         (above? a1 a2))))
\endschemedisplay
Note\slatexdisable{enableslatex} that with the above
settings, \p{lambda} and \p{and} have lost their
default keyword status, ie, they will not be typed
{\bf boldface}.  To retrieve the original status of
special symbols, you should use
\p{\unsetspecialsymbol}, eg,

\verb{
\unsetspecialsymbol{lambda and}
}
\enableslatex
\unsetspecialsymbol{lambda and}

\n Typing the same program after unsetting the
special symbols as above produces, as expected:

\schemedisplay
(define equiv?
  (lambda (a1 a2)
    (and (below? a1 a2) 
         (above? a1 a2))))
\endschemedisplay

In effect, \slatexdisable{enableslatex}
\p{\setspecialsymbol} extends the
basic ``fonting'' capability to arbitrarily special
typeset versions.

\index{schemecasesensitive@{\tt\bs schemecasesensitive}}
\index{case sensitivity}
\re{\p{\schemecasesensitive}}

SLaTeX always typesets output that is of the same case
as your input, regardless of the setting of the
\p{\schemecasesensitive} command.  However, this command
can be used to signal to SLaTeX that all case variations of
an identifier are to be treated identically.  Eg, typing
\p{\schemecasesensitive{false}} ensures
that \p{Lambda}, \p{LAMBDA} and \p{LaMbDa} will
all be treated as keywords, just as \p{lambda}.
\p{\schemecasesensitive{true}} reverts to the
default mode where case is significant in determining
the class of a token.

Note that the status \p{\schemecasesensitive} also
affects the ``special symbols'' of the previous item.
Thus, in the default case-{\it sensitive\/} setting, only the
case-significant symbol as mentioned in the call to
\p{\setspecialsymbol} will be replaced by the
corresponding TeX expansion.  In a case-{\it in\/}sensitive
setting, all case variations of the special symbol will
be replaced.

\index{abovecodeskip@{\tt\bs abovecodeskip}}
\index{belowcodeskip@{\tt\bs belowcodeskip}}
\index{leftcodeskip@{\tt\bs leftcodeskip}}
\index{rightcodeskip@{\tt\bs rightcodeskip}}
\index{schemedisplay@{\tt\{schemedisplay\}}!adjusting display parameters}
\re{\p{\abovecodeskip}
\p{\belowcodeskip}
\p{\leftcodeskip}
\p{\rightcodeskip}}

These are the parameters used by \p{{schemedisplay}} for
positioning  displayed code.  The default values are

\verb{
\abovecodeskip \medskipamount
\belowcodeskip \medskipamount
\leftcodeskip 0pt
\rightcodeskip 0pt
}

\n This produces a flushleft display.  The defaults can be
changed to get new display styles.  Eg, this manual
sets

\verb{
\leftcodeskip\parindent
}

\n which shifts the display from the left by the same
amount as a paragraph indentation.

\index{schemedisplay@{\tt\{schemedisplay\}}!allowing page
breaks in}
\index{schemedisplay@{\tt\{schemedisplay\}}!disallowing
pagebreaks in}

In both the above cases, the \p{{schemedisplay}}
environment will be broken naturally across page
boundaries at the right spot if the code is too long to
fit a single page.  In fact, automatic pagebreaks
within the Scheme code are allowed if and only if
\p{\rightcodeskip} is 0pt (its default value).  For
all other values of \p{\rightcodeskip}, each Scheme
code block in a \p{{schemedisplay}} is guaranteed
to be on the same page.  If you have decided on a left
indentation, and you're not sure what value to give to
\p{\rightcodeskip}, but nevertheless don't want
Scheme code broken across pages, you could set

\verb{
\rightcodeskip=0.01pt %or
\rightcodeskip=0pt plus 1fil
}

\n To understand why this would disable pagebreaks
within the Scheme block, suppose instead you'd set

\verb{
\leftcodeskip=0pt plus 1fil
\rightcodeskip=0pt plus 1fil
}

\n This will get you a {\it centered\/} display style.
This is of course because the skip on each side of the
code produces a ``spring''\f{Springs, or rather
``glue'', are discussed in detail in {\em The
TeXbook}~\cite[pp 70ff]{tex}.}  that 
pushes the code to the center.  But for this spring
action to work nicely, the code must have been
collected into an unbreakable box --- which is
precisely what
\p{{schemedisplay}} does for each of its code blocks
whenever it notices that the prevailing value of
\p{\rightcodeskip} is not the default
zero.\f{0pt plus 1fil $\ne$ 0pt}  Clearly, such
unbreakable boxes cannot tolerate pagebreaks.

Thus, the behind-the-scenes selective boxing dictates
whether a \p{{schemedisplay}} block can or cannot be
broken across a page boundary.  And the value of
\p{\rightcodeskip} is used to govern this selection
in a ``reasonable'' manner.

\index{keywordfont@{\tt\bs keywordfont}}
\index{constantfont@{\tt\bs constantfont}}
\index{variablefont@{\tt\bs variablefont}}
\index{datafont@{\tt\bs datafont}}
\index{specifying SLaTeX's fonts}
\re{\p{\keywordfont}
\p{\constantfont}
\p{\variablefont}
\p{\datafont}}

These decide the typefaces used for keywords,
constants, variables, and data.  The default
definitions are:\f{\p{\sf} in LaTeX is the
sans-serif font.  Since plain TeX does not define
\p{\sf}, you may define one before
\p{\input}ing \p{slatex.sty}.  If no \p{\sf}
is found, \p{slatex.sty} has \p{\let\sf=\rm}.  You
may redefine it before or after to an appropriate
font of your choice.  This manual set
\p{\font\sf=cmss10}.  Of course, you could
redefine \p{\constantfont} itself to not rely on
(the name) \p{\sf}.}

\verb{
\def\keywordfont#1{{\bf#1}}
\def\constantfont#1{{\sf#1}}
\def\variablefont#1{{\it#1\/}}
\let\datafont\constantfont
}

\n This is close to the {\em Little
Schemer}~\cite{tls,tss} style.
Redefine these control 
sequences for font changes.  As an extreme case, defining
all of them to
\p|{{\tt#1}}| typesets everything in monospace
typewriter font, as, for instance, in SICP~\cite{sicp}.

Note that ``constants'' and ``data'' {\it can\/} be
distinguished, although by default SLaTeX does not do
so.  Typically, primitive data such as booleans,
numbers, characters and strings are set as ``data'';
whereas quoted material is set as ``constant''.

The control sequences \p{\keywordfont}, \&c., can be
defined to be anything at all, not just font switches.
For instance, if you use  Rokicki's dvips, you can
use these SLaTeX sequences to color-code your programs!
Simply use:

\begingroup
\def\keywordfont#1{\Red{#1}}
\def\variablefont#1{\Blue{#1}}
\def\constantfont#1{\Green{#1}}

\verb{
\input colordvi
\def\keywordfont#1{\Red{#1}}
\def\variablefont#1{\Blue{#1}}
\def\constantfont#1{\Green{#1}}
}

\n The following example will appear in color if
you've processed this document with dvips and have a
color-capable viewer or printer:

\enableslatex
\schemedisplay
(define factorial
  (lambda (n)
    (if (= n 0) 1
        (* n (+ n 1)))))
\endschemedisplay

\endgroup\slatexdisable{enableslatex}

\index{defschemedisplaytoken@{\tt\bs defschemedisplaytoken}}
\index{defschemetoken@{\tt\bs defschemetoken}}
\index{defschemeboxtoken@{\tt\bs defschemeboxtoken}}
\index{defschemeresulttoken@{\tt\bs defschemeresulttoken}}
\index{defschemeresponsetoken@{\tt\bs defschemeresponsetoken}}
\index{defschemeresponseboxtoken@{\tt\bs defschemeresponseboxtoken}}
\index{defschemeinputtoken@{\tt\bs defschemeinputtoken}}
\index{defschemeregiontoken@{\tt\bs defschemeregiontoken}}
\index{defining SLaTeX control sequences}
\re{\p{\defschemedisplaytoken}
\p{\defschemetoken}
\p{\defschemeboxtoken}
\p{\defschemeresulttoken}
\p{\defschemeresponsetoken}
\p{\defschemeresponseboxtoken}
\p{\defschemeinputtoken}
\p{\defschemeregiontoken}}

These define the tokens used by SLaTeX to trigger
typesetting of in-text code, displayed code, boxed
code, Scheme program files and robust regions.  The
default tokens are, as already described,
\p{{schemedisplay}}, \p{\scheme},
\p{{schemebox}}, \p{\schemeresult},
\p{{schemeresponse}}, \p{{schemeresponsebox}},
\p{\schemeinput} and \p{{schemeregion}}
respectively.   You can use the \p{\defscheme*token}
control sequences to get alternate tokens, eg,
shorter or more mnemonic ones.  Thus,
if you want \p{\code} to be
your new control sequence for in-text code, use
\p|\defschemetoken{code}|.  All instances of
\p{\code+...+} after this definition produce
in-text code, unless overridden by an
\p{\undefschemetoken} command.

One can have at any time any number of tokens for the
same activity.  One consequence of this is that one can
have nested \p{{schemeregion}}s, provided one has
different names for the nested call.  Otherwise, the
\p{\end} of an inner region will prematurely
terminate an outer region.

\index{undefschemedisplaytoken@{\tt\bs undefschemedisplaytoken}}
\index{undefschemetoken@{\tt\bs undefschemetoken}}
\index{undefschemeboxtoken@{\tt\bs undefschemeboxtoken}}
\index{undefschemeresulttoken@{\tt\bs undefschemeresulttoken}}
\index{undefschemeresponsetoken@{\tt\bs undefschemeresponsetoken}}
\index{undefschemeresponseboxtoken@{\tt\bs undefschemeresponseboxtoken}}
\index{undefschemeinputtoken@{\tt\bs undefschemeinputtoken}}
\index{undefschemeregiontoken@{\tt\bs undefschemeregiontoken}}
\index{undefining SLaTeX control sequences}
\re{\p{\undefschemedisplaytoken}
\p{\undefschemetoken}
\p{\undefschemeboxtoken}
\p{\undefschemeresulttoken}
\p{\undefschemeresponsetoken}
\p{\undefschemeresponseboxtoken}
\p{\undefschemeinputtoken}
\p{\undefschemeregiontoken}}

These {\it un\/}define the tokens used for triggering
the typesetting of in-text code, displayed code, boxed
code, Scheme program files, and robust regions.
Typically, tokens are undefined so you can use the
names for other purposes without tripping up the SLaTeX
system.

\index{defschememathescape@{\tt\bs defschememathescape}}
\index{undefschememathescape@{\tt\bs undefschememathescape}}
\index{TeX mathmode in SLaTeX}
\index{escape character for mathmode within Scheme}
\re{\p{\defschememathescape}
\p{\undefschememathescape}}

{\tolerance100000
\p|\defschememathescape{$}| defines the character
\p{$} as a mathematical escape character to be used
within Scheme code.  (Any character other than
\p|}| and whitespace may be chosen instead of
\p{$}.)  This allows one to use TeX
mathematical subformulas within Scheme code, eg,\par}

\verb{
\defschememathescape{$}

\begin{schemedisplay}
(define $\equiv$
  (lambda (a$_1$ a$_2$)
    ($\land$
      ($\subseteq$ a$_1$ a$_2$)
      ($\supseteq$ a$_1$ a$_2$))))
\end{schemedisplay}
}

\n produces

\enableslatex
\defschememathescape{$}

\schemedisplay
(define $\equiv$
  (lambda (a$_1$ a$_2$)
    ($\land$
      ($\subseteq$ a$_1$ a$_2$)
      ($\supseteq$ a$_1$ a$_2$))))
\endschemedisplay
\undefschememathescape{$}
\slatexdisable{enableslatex}
\p|\undefschememathescape{$}| disables the
math-escape nature, if any, of \p{$}.

\index{slatexdisable@{\tt\bs slatexdisable}}
\index{disabling SLaTeX}
\re{\p{\slatexdisable}}

The tokens for typesetting code, as also the token
\p{\input} (which is sensitive to SLaTeX, since
the latter uses it to recursively process files within
files), can only be used as calls.  If they occur in
the bodies of macro definitions, or their names are
used for defining other control sequences, SLaTeX could
misprocess them.  Sometimes, one wants TeX to
\p{\input} a file, without wanting SLaTeX to process
the inputted file.  Or the name
\p{\scheme} can occur in a verbatim environment,
and we don't want such an occurrence to cause SLaTeX to
look for and ``find'' Scheme code that is not really
there.

Avoiding such uses altogether can be unduly
restrictive.\f{Especially when one is writing
a ``How to \dots'' manual like this where one both uses
{\it and\/} mentions the control sequences!} One way
out is to judiciously employ the
\p{\undefscheme*token} commands to temporarily
remove the SLaTeX-specificity of these names.  Even
this can be painful.  SLaTeX therefore provides the
command \p{\slatexdisable}.  This takes one
argument word and makes the corresponding control
sequence out of it.  Further, from this point in the
text, SLaTeX is disabled {\it until\/} the
manufactured control sequence shows up.  This
mechanism makes it possible to restrict SLaTeX to only
appropriate portions of the text.  Note that the token
\p{\slatexdisable} itself can appear in the text
succeeding its call.  The only token that can restore
SLaTeX-sensitivity is the one created during the call
to \p{\slatexdisable}.

The following is a typical example of the
\p{\slatexdisable} approach.  You want the names
\p{\scheme} and
\p|\begin{schemedisplay}| in a {\tt verbatim}
environment:

\verb{
\slatexdisable{slatexenable}
\begin{verbatim}
SLaTeX provides the command \scheme
and the pair \begin{schemedisplay}
and \end{schemedisplay} to typeset
in-text and displayed Scheme code
respectively.
\end{verbatim}
\slatexenable
}

\n produces the required

\verb{
SLaTeX provides the command \scheme
and the pair \begin{schemedisplay}
and \end{schemedisplay} to typeset
in-text and displayed Scheme code
respectively.
}

\index{slatexignorecurrentfile@{\tt\bs slatexignorecurrentfile}}
\index{disabling SLaTeX}
\re{\p{\slatexignorecurrentfile}}

This is a SLaTeX pragma included to improve efficiency.
If you're sure that the remaining portion of a certain
(La)TeX file (including the files that would be
\p{\input}ed by it) don't contain any SLaTeX
commands, then you may place this control sequence in
it at this point to signal SLaTeX that no preprocessing
is necessary for the rest of the file.

\index{defslatexenvstyle@{\tt\bs defslatexenvstyle}}
\index{plain TeX}
\index{LaTeX}
\index{environments}
\re{\p{\defslatexenvstyle}}

As shown previously, the differences in SLaTeX usage
between plain TeX and LaTeX is simply a matter of the
difference in the ``environment'' styles of the two
formats.  It is easy get the behavior of the one format
with the other.

1.  If you wish to use the plain-TeX style in LaTeX,
type

\verb{
\defslatexenvstyle{tex}
}

\n before first such use.

2. Similarly, if you wish to use the LaTeX
\p{\begin}/\p{\end} style in plain TeX, use

\verb{
\defslatexenvstyle{latex}
}

\n {\it provided you have already defined \p{\begin} and
\p{\end} appropriately!\/}  One way to accomplish
this is:

\verb{
\def\begin#1{\begingroup
  \csname#1\endcsname
  \let\end\endenvironment}
\def\endenvironment#1{%
  \csname end#1\endcsname
  \endgroup}
}

\n Here, \p{\end} is defined within a group because
TeX already has an \p{\end} command whose global
definition is used by commands such as \p{\bye} and
therefore should not be changed lightly.

{\tolerance100000
In either case, you can revert to the default style with
\p|\defslatexenvstyle{latex}| and
\p|\defslatexenvstyle{tex}|
respectively.\par}

\index{slatexseparateincludes@{\tt\bs slatexseparateincludes}}
\index{reusing SLaTeX's temporary files}
\re{\p{\slatexseparateincludes}}

By default, the temporary files of SLaTeX use the name
of the topmost TeX file, ie, the name stored under
\p{\jobname}.  In large LaTeX documents using
\p{\include}, this may be unduly restrictive.

To recapitulate, the \p{slatex} command creates
temporary files to store typeset code and then passes
the baton on to TeX or LaTeX.  If no significant change
has been made to the Scheme code (either in content or
in relative positioning) in the document, then
successive calls to (La)TeX could be made directly
using the old temporary files.  This could be a time-saver,
since it avoids calling up the Scheme typesetter.

However, in a large LaTeX document with
\p{\include}s, these successive calls to LaTeX often
entail juggling the \p{\include}s that are chosen.
In this case, even though the relative position of the
Scheme code is preserved within each \p{\include}d
file, the sequence perceived by the main file changes.
This spoils the invariance we needed if we'd wanted to
avoid calling SLaTeX unnecessarily.

\index{reusing SLaTeX's temporary files!exploiting
LaTeX's {\tt\bs include}}

To solve this, the SLaTeX command sequence
\p{\slatexseparateincludes} --- which must be called
before the first occurrence of Scheme code in your
document --- guarantees that each
\p{\include}d file will generate its own pool of
temp files.  Thus, if the SLaTeX
files are created once for each \p{\include}, they
will be correctly loaded no matter what sequence of
\p{\include}s is taken.

\index{schemecodehook@{\tt\bs schemecodehook}}
\index{hook for {\tt\bs schemedisplay} and
{\tt\bs schemebox}}
\re{\p{\schemecodehook}}

The user can define \p{\schemecodehook} to be
anything.  The hook will be evaluated inside each
subsequent call to \p{{schemedisplay}} and
\p{{schemebox}}.  Eg,  if you have \p{\tiny}
defined (as in LaTeX) to reduce font size, then

\verb{
\let\schemecodehook\tiny
}

\n converts your Scheme displays and boxes into {\fiverm
small print}.

The default value of the hook is \p{\relax}, a
no-op.

\beginchapter 4 Resetting SLaTeX's defaults

\label{preamble}
\index{writing personal preamble}
\index{SLaTeX database!modifying}
{\tolerance100000 A sample style modification file for
SLaTeX would include redefinition of the names of the
codesetting control sequences; adjustment of the
display parameters; modification of the font
assignments for keywords, constants, variables, and
special symbols; and addition of new keywords,
constants, variables, and special symbols to SLaTeX's
database.\par}

Let's assume you want

1. a centered display style with no vertical skips;

{\tolerance100000 2.  the names \p{\code},
\p|{schemefrag}|,
\p|{scmbox}|, \p{\sinput} instead of
\p{\scheme}, \p{{schemedisplay}},
\p{{schemebox}} and
\p{\schemeinput};\par}

3. tokens to disregard case;

4. the keywords to come out in \p{typewriter}, the
constants in roman, and the variables in {\bf bold};

5. \p{und} and \p{oder} as keywords,
\p{true} and \p{false} as constants,
\p{define} as a variable (overriding default as
keyword!), \p{F} as a constant (\p{f} will also
be a constant, due to case-insensitivity!);

6. \p{top} and \p{bottom} to print as
$\top$ and $\bot$ respectively.

This could be set up as

\verb{
\abovecodeskip 0pt
\belowcodeskip 0pt
\leftcodeskip 0pt plus 1fil
\rightcodeskip 0pt plus 1fil

\undefschemetoken{scheme}
\undefschemeboxtoken{schemebox}
\undefschemedisplaytoken{schemedisplay}
\undefschemeinputtoken{schemeinput}

\defschemetoken{code}
\defschemeboxtoken{scmbox}
\defschemedisplaytoken{schemefrag}
\defschemeinputtoken{sinput}

\schemecasesensitive{false}

\def\keywordfont#1{{\tt#1}}
\def\constantfont#1{{\rm#1}}
\def\variablefont#1{{\bf#1\/}}

\setkeyword{und oder}
\setconstant{true false}
\setvariable{define}
\setconstant{F}

\setspecialsymbol{top}{$\top$}
\setspecialsymbol{bottom}{$\bot$}
}

\n This file can then be \p{\input} in the preamble of
your (La)TeX document.

\beginchapter 5 Obtaining and installing SLaTeX

\label{download}
\index{obtaining and installing SLaTeX}
SLaTeX is available as a \p{.tar.bz2} package from Northeastern University at
the URL \url{http://www.ccs.neu.edu/~dorai/slatex/slatxdoc.html}.
\p{bunzip2}'ing and
un\p{tar}'ing produces a directory \p{slatex},
containing, among others, the files \p{slatex.sty} and
\p{scmxlate-slatex-src.scm}. 

To configure SLaTeX on your system:

1. Copy (or move, or link) \p{slatex.sty} to a suitable place, ie,
one of the directories mentioned in your \p{TEXINPUTS} variable.

2. Ensure that you have
\urlp{scmxlate}{http://www.ccs.neu.edu/~dorai/scmxlate/scmxlate.html}
installed on your system.  Note down the full pathname
of the file \p{scmxlate.scm} file in the \p{scmxlate}
directory as unpacked on your system.

 
3. Change directory (\p{cd}) to \p{slatex}.\f{The
SLaTeX files use Unix-style newlines.  If you're on a
Windows box, you may want to use an appropriate newline
modifier to make the files comply with your operating
system's newline format.}

3. Edit the file \p{scmxlate-slatex-src.scm} as suggested by the
comments in the file itself.  In particular, the filename 
in the expression

\enableslatex
\schemedisplay
(scmxlate-target-file "/home/dorai/.www/slatex/slatex.scm")
\endschemedisplay
should be changed to reflect where {\em you} will
(manually) place the \p{slatex.scm} file that will be created by
the configuration process. 

4. Invoke your Scheme or Common Lisp interpreter.
Load the file \p{scmxlate/scmxlate.scm}, using its
correct relative or full pathname.  Eg,

\schemedisplay
(load "wherever-it-is/scmxlate/scmxlate.scm")
\endschemedisplay
at the Scheme (or Common
Lisp) prompt.  You may be asked some questions.
\p{scmxlate} will configure SLaTeX for your
Scheme dialect and operating system, creating a Scheme
file called \p{slatex.scm}.  

The configuration process also creates
a shell script \p{slatex} (\p{slatex.bat}
on Windows) and a Scheme file \p{callsla.scm}.
The shellscript lets you invoke SLaTeX from your operating-system command
line.  The file \p{callsla.scm} lets you call SLaTeX 
from Scheme or Common Lisp.  The latter is a convenient alternative 
if the shellscript cannot be created for your system, or
if you prefer working within a Scheme or Common Lisp
session anyway.

5. Exit Scheme or Common Lisp (if \p{scmxlate} hasn't
exited on its own).  Copy (or move, or link) the 
newly created \p{slatex.scm} to the location you identified
in \p{scmxlate-slatex-src.scm}.  Copy (or move, or link)
the shell script \p{slatex} or \p{slatex.bat} to a directory
in your \p{PATH}.  Copy (or move, or link) the file
\p{callsla.scm} to a place convenient to you or
your Scheme/CL system.

If you have the shellscript, and if the document 
in question is \p{slatxdoc.tex},
SLaTeX can be invoked
as

\p{
slatex slatxdoc.tex
}

\n (The extension is optional if it is \p{.tex}.)

Using the \p{callsla.scm}, the same can be accomplished
by loading \p{callsla.scm} into your Scheme/CL and 
then typing

\schemedisplay
(call-slatex "slatxdoc.tex")
\endschemedisplay
(Again, the extension \p{.tex} is optional.)

(Yes, \p{slatxdoc.tex} is the document you're reading
now.  You might as well check your configured SLaTeX on
its own documentation.)

SLaTeX has been tested on many
Scheme and Common Lisp implementations.  If your
dialect doesn't work, send email to \p{ds26} at \p{gte}
dot \p{com}, with details about your Scheme's macro
facility, and how it implements the non-standard Scheme
procedures \scheme{delete-file}, \scheme{file-exists?},
and \scheme{force-output}.

\beginchapter 6 References

\bibliographystyle{plain}

\bibliography{slatxdoc}

\beginchapter 7 Index

%\begincolumns2
\inputindex
%\endcolumns

\bye
