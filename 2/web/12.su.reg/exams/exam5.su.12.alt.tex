\documentclass[11pt]{article}
\usepackage{amsmath}
%% ------------------------------------------------------------------
%% SOLUTIONS:
\def\thel{\noindent\rule{2.5cm}{.5pt}}
\long\def\begsol#1 #2\endsol{#1}
%\def\begsol#1{\thel {\bf Solution} \thel}\def\endsol{\relax}
%%uncomment the above line to get the solutions printed

\newcommand\code[1]{\texttt{#1}}
\newcommand\bcode[1]{\texttt{\textbf{#1}}}

%% PROBLEMS:
\def\pts#1{\marginpar{\footnotesize \raggedright  \fbox{#1 {\sc Points}}}}
\newcounter{Pctr}
\newenvironment{problem}{\stepcounter{Pctr}%
\begin{description}
\item[\noindent{\bf Problem} \arabic{Pctr}] 
\end{description}}{\relax}
%% ------------------------------------------------------------------

%% SUBPARTS:
\newcounter{parts}
\renewcommand{\theparts}{\Alph}
%% ------------------------------------------------------------------

\begin{document}

\renewcommand{\theenumi}{\Alph{enumi}}
\setcounter{Pctr}{0}

%% ------------------------------------------------------------------

\vfill
\centerline{\Large CS 2510 Exam 5 (Alternate) -- Summer 2012}

~\\[2cm]

\begin{center}
\begin{tabular}{l@{\qquad}l}
Name:                        & \rule{200pt}{.1pt} \\[.5cm]
Student Id (last 4 digits):  & \rule{200pt}{.1pt} \\[.5cm]
\end{tabular}
\end{center}

\noindent\begin{minipage}{7.5cm} $\bullet$ Write down the answers in the
space provided. 

$\bullet$ You may use all syntax of Java that we have studied in
class.

$\bullet$ For tests you only need to provide the expression that
computes the actual value, connecting it with an arrow to the expected
value. For example \code{s.method() -> true} is sufficient.

$\bullet$ Remember that the phrase ``design a class'' or ``design a
method'' means more than just providing a definition. It means to
design them according to the \textbf{design recipe}.  You are
\textit{not} required to provide a method template unless the problem
specifically asks for one.  However, be prepared to struggle if you
choose to skip the template step.

\bigskip

\textit{Good luck!}
\end{minipage}\hfil\begin{minipage}[t]{4.5cm}
\begin{tabular}{|c|l@{\qquad\qquad}|r|}
\hline
\textbf{Score} &  & 45 \\ \hline
\end{tabular}
\end{minipage}

\vfill\thispagestyle{empty}
\newpage

%% -----------------------------------------------------------------------------
%% 
\pts{45}
\begin{problem}
Develop an implementation of {\tt Comparator<Square>} that orders
squares by their area in such a way that the smaller the area, the
\emph{greater} the square is considered.  So for example, there is no
square greater than the one with width $= 0$ and height $= 0$.

You may rely on the following definition of {\tt Square}:

\begin{verbatim}
// Represents a square with given height and width.
class Square {
  Integer width;
  Integer height;
  Square(Integer width, Integer height) {
    this.width = width;
    this.height = height;
  }
}
\end{verbatim}
\end{problem}

\newpage
\begin{problem}
Design a method:
\begin{center}
{\tt <T> Boolean noSameNeighbors(ArrayList<T> ls, Comparator<T> c)}
\end{center}
that determines if the given array list has no adjacent elements
(i.e.~neighboring elements) that are considered of equal size
according to the given comparator.
\end{problem}
\newpage
\begin{problem}
Design a method:
\begin{verbatim}
<A> Comparator<ArrayList<A>> dict(Comparator<A> ca)
\end{verbatim}
that consumes a comparator for {\tt A}s and and produces a comparator
for array lists of {\tt A}s that is the \emph{dictionary order} (aka
\emph{alphabetic order}).

In other words, given an ordering on individual elements, produce an
ordering on \emph{lists of} elements that corresponds to the ordering
used in dictionaries.  A dictionary uses an ordering on individual
letters (single elements) to make an ordering on strings of letters
(lists of elements) so that ``abc'' $<$ ``abcd'', ``abc'' $<$ ``adc'',
``abc'' $>$ ``ab'', and ``abc'' $=$ ``abc'', etc.
\end{problem}
\ 
\newpage
\ 
\newpage
\ 
\end{document}

