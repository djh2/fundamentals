\documentclass[11pt]{article}

%% ------------------------------------------------------------------
%% SOLUTIONS:
\def\thel{\noindent\rule{2.5cm}{.5pt}}
\long\def\begsol#1 #2\endsol{#1}
%\def\begsol#1{\thel {\bf Solution} \thel}\def\endsol{\relax}
%%uncomment the above line to get the solutions printed

\newcommand\code[1]{\texttt{#1}}
\newcommand\bcode[1]{\texttt{\textbf{#1}}}

%% PROBLEMS:
\def\pts#1{\marginpar{\footnotesize \raggedright  \fbox{#1 {\sc Points}}}}
\newcounter{Pctr}
\newenvironment{problem}{\stepcounter{Pctr}%
\begin{description}
\item[\noindent{\bf Problem} \arabic{Pctr}] 
\end{description}}{\relax}
%% ------------------------------------------------------------------

%% SUBPARTS:
\newcounter{parts}
\renewcommand{\theparts}{\Alph}
%% ------------------------------------------------------------------

\begin{document}

\renewcommand{\theenumi}{\Alph{enumi}}
\setcounter{Pctr}{0}

%% ------------------------------------------------------------------

\vfill
\centerline{\Large CS 2510 Exam 1 -- Spring 2012}

~\\[2cm]

\begin{center}
\begin{tabular}{l@{\qquad}l}
Name:                        & \rule{200pt}{.1pt} \\[.5cm]
Student Id (last 4 digits):  & \rule{200pt}{.1pt} \\[.5cm]
\end{tabular}
\end{center}

\noindent\begin{minipage}{7.5cm} $\bullet$ Write down the answers in the
space provided. 

$\bullet$ You may use all syntax that you know from \textit{FunJava}
(that is, the parts of Java we have studied in class), although there
are several features you will \emph{not} need.

$\bullet$ When defining methods, you do not need to give a complete
class definition---just indicate in which class your method definition
should be placed.

$\bullet$ For tests you only need to provide the expression that
computes the actual value, connecting it with an arrow to the expected
value. For example \code{s.method() -> true} is sufficient.

$\bullet$ Remember that the phrase ``design a class'' or ``design a
method'' means more than just providing a definition. It means to
design them according to the \textbf{design recipe}.  You are
\textit{not} required to provide a method template unless the problem
specifically asks for one.  However, be prepared to struggle if you
choose to skip the template step.

$\bullet$ We will not answer \textit{any} questions during the exam.

\bigskip

\textit{Good luck!}
\end{minipage}\hfil\begin{minipage}[t]{4.5cm}
\begin{tabular}{|c|l|@/r|}
\hline
\textbf{Problem} & Points & \\ \hline
A & & 3\\ \hline
B & & 9\\ \hline
C & & 9\\ \hline
D & & 9\\ \hline
E & & 4\\ \hline
\textbf{Total} &  & 34 \\ \hline
\end{tabular}
\end{minipage}

\vfill\thispagestyle{empty}
\newpage

%% -----------------------------------------------------------------------------
%% 
\pts{34}
\begin{problem}  Valentine's day is coming, so here is a nice gift you
  can give or receive. Of course, it needs to be in a nice box, and
  wrapped carefully.

Here is a class diagram that describes this gift:

\begin{verbatim}
                     +--------------+
                     |     IGift    |<------------------+
                     +--------------+                   |
                     / \   / \   / \                    |
                     ---   ---   ---                    |
                      |     |     |                     |
        +-------------+     |     +-----------+         |
        |                   |                 |         |
        |                   |                 |         |
 +-------------+      +--------------+  +------------+  |
 | Jewel       |<--+  | Box          |  | Wrap       |  |
 +-------------+   |  +--------------+  +------------+  |
 | String name |   |  | String kind  |  | IGift gift |--+ 
 +-------------+   +--| Jewel inside |  +------------+
                      +--------------+ 
\end{verbatim}


\begin{enumerate}


\newpage
%-----------------------------------------------
\item {\em{(3 points)}}

 Make examples of data (i.e., instances) for this class
    hierarchy. Include an example of a ruby in a red box with three
    wrappings around it.

\begsol{\vspace{0.5in}}
[POINTS 3: 1 point for the \texttt{Jewel} example, 1 point for the a 
\texttt{Box}, 1 point for the the specified example of \texttt{Wrap}] 

%{\vspace{0.5in}}
\begin{verbatim}
IBang p0 = new Pop(0);
IBang p1 = new Pop(1);
IBang p2 = new Pop(2);
IBang p3 = new Pop(3);
	
IBang b0 = new Fizz(4, this.p0, this.p0);
IBang b1 = new Fizz(4, this.p2, this.p3);
IBang b1t1 = new Fizz(5, this.p3, this.p3);
IBang b1t2 = new Fizz(6, this.p3, this.b0);
IBang b1t7 = new Fizz(7, this.b0, this.b0);
\end{verbatim}
\endsol


\newpage
%-----------------------------------------------
\item {\em{(9 points)}}

Design a method, \code{size}, that computes the size of the gift
represented by a \code{IGift}.

      The size of \code{Jewel} is always \code{5}.  The size
      of a \code{Box} is \code{8} regardless of what it contains. Each
      wrap contributes one to the total size.


\begsol{\vspace{0.5in}}
[POINTS 9: 1 point for the purpose and header in the interface
\texttt{IGift}; in the class \texttt{Jewel}: 1 point for the body, 1
point for tests; in the class \texttt{Box}: 1 point for the body, 1
point for tests; in the class \texttt{Wrap}: 2 
points for the body, 2 points for examples]

%{\vspace{0.5in}}
\begin{verbatim}
// in the interface IBang:

// compute the age of this pop
int howOld();

// in the class Pop:
// compute the age of this pop
int howOld(){
  return this.age;
}

// in the class Fizz:
// compute the age of this fizz
int howOld(){
  return this.age + 
         Math.max(this.bang1.howOld(), 
                  this.bang2.howOld());
}

// in the Examples class
// test the method howOld for the IBang classes
boolean testHowOld(Tester t){
  return
  t.checkExpect(this.p0.howOld(), 0) &&
  t.checkExpect(this.p3.howOld(), 3) &&
  t.checkExpect(this.b1.howOld(), 7) &&
  t.checkExpect(this.b1t1.howOld(), 8) &&
  t.checkExpect(this.b1t7.howOld(), 11);
}	
\end{verbatim}
\endsol

\newpage
%-----------------------------------------------
\item {\em{(9 points)}}

Design the method \code{replaceBox} that produces an
    \code{IGift} wrapped in the same wrappings as this one,
    containing the same jewel  inside but with the box replaced by the
    box of the given kind.


\begsol{\vspace{0.5in}}
[POINTS 9: 1 point for the purpose and header in the interface
\texttt{IGift}; in the class \texttt{Jewel}: 1 point for the body, 1
point for tests; in the class \texttt{Box}: 1 point for the body, 1
point for tests; in the class \texttt{Wrap}: 2 
points for the body, 2 points for examples]

%{\vspace{0.5in}}
\begin{verbatim}
// in the interface IBang:

// produce a new bang from this Bang after one tick
IBang nextYear();

// in the class Pop:
// produce a new bang from this Pop after one tick
IBang nextYear(){
  if (this.age == 3){
    return new Fizz(4, new Pop(0), new Pop(0));
  } else {
    return new Pop(this.age + 1);
  }
}

// in the class Fizz:
// produce a new bang from this Fizz after one tick
IBang nextYear(){
  if (this.bang2.howOld() > this.bang1.howOld()){
    return new Fizz(this.age + 1, 
             this.bang1.nextYear(), this.bang2);
  } else {
    return new Fizz(this.age + 1, 
             this.bang1, this.bang2.nextYear());
  }
}

// in the Examples class:
// test the method nextYear for the IBang classes
boolean testNextYear(Tester t){
  return
  t.checkExpect(this.p0.nextYear(), this.p1) &&
  t.checkExpect(this.p3.nextYear(), this.b0) &&
  t.checkExpect(this.b1.nextYear(), this.b1t1) &&
  t.checkExpect(this.b1t1.nextYear(), this.b1t2) &&
 t.checkExpect(this.b1t2.nextYear(), this.b1t7);
}
\end{verbatim}
\endsol

\newpage
%-----------------------------------------------
\item {\em{(9 points)}}

Design the method \code{sameJewelBox} that checks whether this gift
contains the jewel with the given name and is inside of the box of the
given kind.


\begsol{\vspace{0.5in}}
[POINTS 9: 1 point for the purpose and header in the interface
\texttt{IGift}; in the class \texttt{Jewel}: 1 point for the body, 1
point for tests; in the class \texttt{Box}: 1 point for the body, 1
point for tests; in the class \texttt{Wrap}: 2 
points for the body, 2 points for examples]

%{\vspace{0.5in}}
\begin{verbatim}
// in the interface IBang:

// produce a new bang from this Bang after one tick
IBang nextYear();

// in the class Pop:
// produce a new bang from this Pop after one tick
IBang nextYear(){
  if (this.age == 3){
    return new Fizz(4, new Pop(0), new Pop(0));
  } else {
    return new Pop(this.age + 1);
  }
}

// in the class Fizz:
// produce a new bang from this Fizz after one tick
IBang nextYear(){
  if (this.bang2.howOld() > this.bang1.howOld()){
    return new Fizz(this.age + 1, 
             this.bang1.nextYear(), this.bang2);
  } else {
    return new Fizz(this.age + 1, 
             this.bang1, this.bang2.nextYear());
  }
}

// in the Examples class:
// test the method nextYear for the IBang classes
boolean testNextYear(Tester t){
  return
  t.checkExpect(this.p0.nextYear(), this.p1) &&
  t.checkExpect(this.p3.nextYear(), this.b0) &&
  t.checkExpect(this.b1.nextYear(), this.b1t1) &&
  t.checkExpect(this.b1t1.nextYear(), this.b1t2) &&
 t.checkExpect(this.b1t2.nextYear(), this.b1t7);
}
\end{verbatim}
\endsol

\newpage
%-----------------------------------------------
{\center{... This page intentionally left blank ...}}


\newpage
%-----------------------------------------------
\item {\em{(4 points)}}

Show the resulting templates for the \code{Box} class.

\begsol{\vspace{0.5in}}
[POINTS 4:
  Pop -- field and two methods with types  -- 1 point;
  Bing -- fields with types                -- 1 point;
          methods defined here with types  -- 1 point;
          methods for fields with types    -- 1 point]]

%{\vspace{0.5in}}
\begin{verbatim}
// in the class Pop:
/* Template:
   Fields:
   ... this.age ...     -- int
   
   Methods:
   ... this.howOld() ...   -- int
   ... this.nextYear() ... -- IBang
*/

// in the class Fizz:
/* Template:
   Fields:
   ... this.age ...     -- int
   ... this.bang1 ...     -- IBang
   ... this.bang2 ...     -- IBang
   
   Methods:
   ... this.howOld() ...   -- int
   ... this.nextYear() ... -- IBang
   
   Methods for fields:
   ... this.bang1.howOld() ...   -- int
   ... this.bang2.howOld() ...   -- int
   ... this.bang1.nextYear() ... -- IBang
   ... this.bang2.nextYear() ... -- IBang   
 */	
\end{verbatim}
\endsol


\end{enumerate}
\end{problem}
\end{document}

