\documentclass[11pt]{article}

%% ------------------------------------------------------------------
%% SOLUTIONS:
\def\thel{\noindent\rule{2.5cm}{.5pt}}
\long\def\begsol#1 #2\endsol{#1}
%\def\begsol#1{\thel {\bf Solution} \thel}\def\endsol{\relax}
%%uncomment the above line to get the solutions printed

\newcommand\code[1]{\texttt{#1}}
\newcommand\bcode[1]{\texttt{\textbf{#1}}}

%% PROBLEMS:
\def\pts#1{\marginpar{\footnotesize \raggedright  \fbox{#1 {\sc Points}}}}
\newcounter{Pctr}
\newenvironment{problem}{\stepcounter{Pctr}%
\begin{description}
\item[\noindent{\bf Problem} \arabic{Pctr}] 
\end{description}}{\relax}
%% ------------------------------------------------------------------

%% SUBPARTS:
\newcounter{parts}
\renewcommand{\theparts}{\Alph}
%% ------------------------------------------------------------------

\begin{document}

\renewcommand{\theenumi}{\Alph{enumi}}
\setcounter{Pctr}{0}

%% ------------------------------------------------------------------

\vfill
\centerline{\Large CS 2510 Exam 2 -- Summer 2012}

~\\[2cm]

\begin{center}
\begin{tabular}{l@{\qquad}l}
Name:                        & \rule{200pt}{.1pt} \\[.5cm]
Student Id (last 4 digits):  & \rule{200pt}{.1pt} \\[.5cm]
\end{tabular}
\end{center}

\noindent\begin{minipage}{7.5cm} $\bullet$ Write down the answers in the
space provided. 

$\bullet$ You may use all syntax of Java that we have studied in
class.

$\bullet$ For tests you only need to provide the expression that
computes the actual value, connecting it with an arrow to the expected
value. For example \code{s.method() -> true} is sufficient.

$\bullet$ Remember that the phrase ``design a class'' or ``design a
method'' means more than just providing a definition. It means to
design them according to the \textbf{design recipe}.  You are
\textit{not} required to provide a method template unless the problem
specifically asks for one.  However, be prepared to struggle if you
choose to skip the template step.

\bigskip

\textit{Good luck!}
\end{minipage}\hfil\begin{minipage}[t]{4.5cm}
\begin{tabular}{|c|l@{\qquad\qquad}|r|}
\hline
\textbf{Score} &  & 30 \\ \hline
\end{tabular}
\end{minipage}

\vfill\thispagestyle{empty}
\newpage

%% -----------------------------------------------------------------------------
%% 
\pts{30}
\begin{problem}

Binary trees are one of the most widely used data structures in
computer science.  Here are three examples of binary trees of numbers:

\begin{verbatim}
         5        1           3
                   \         / \
                    2       4  18
                              /  \
                             9    12
                            /    /
                           7    3
\end{verbatim}

Nodes in a binary tree carry data, which in this case are numbers, and
a node has a left and right subtree.  Binary trees, like lists, may
contain \emph{any} number of elements, including 5, 8, 19, and 0.

\begin{enumerate}
\item Design a data definition for binary trees.  It must adequately
  represent any possible binary tree.

\item Give class diagrams corresponding to the data definition from
  part A.

\item Design a {\tt mirror} method that computes the mirror image of a
  binary tree.  For example, the mirror image of the above trees would
  be:

\begin{verbatim}
     5        1           3
             /           / \
            2           18  4
                       /  \
                     12    9
                       \    \
                        3    7
\end{verbatim}
\end{enumerate}


\end{problem}
\newpage
\ 
\end{document}

