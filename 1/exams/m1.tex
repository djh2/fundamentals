\documentclass[12pt]{article}
\usepackage{fullpage}
\usepackage{times}
\usepackage{xcolor,censor}
\usepackage{fontspec}
\usepackage{fancyvrb}

\setmainfont{Times New Roman}

\censorruleheight=0ex
\StopCensoring

\begin{document}

\begin{center}
  {\Large {\bf CMSC 131A, Midterm 1}\\
    \censor{{\bf SOLUTION}} \ \\
    \ \\
    Fall 2018 \ \\
  }
\end{center}

\vspace{2em}

\noindent
NAME:\verb|________________________________________________________|

\vspace{2em}

\noindent
UID: \verb|__________________________________________|

\vspace{2em}

\begin{center}
\begin{tabular}{| c | c |}
  \hline
  Question & Points \\ \hline \hline
  1 & 10 \\ \hline
  2 & 10 \\ \hline
  3 & 15 \\ \hline
  4 & 15 \\ \hline
  5 & 40 \\ \hline
  Total: & 90\\
  \hline
\end{tabular}
\end{center}

\noindent
This test is open-book, open-notes, but you may not use any computing
device other than your brain and may not communicate with anyone.
You have 50 minutes to complete the test.

\vskip 1em

\noindent
The phrase ``design a program'' or ``design a function'' means follow
the steps of the design recipe.  Unless specifically asked for, you do
not need to provide intermediate products like templates or stubs,
though they may be useful to help you construct correct solutions.

\vskip 1em

\noindent
You may use any of the data definitions given to you within this exam
and do not need to repeat their definitions.

\vskip 1em

\noindent
Unless specifically instructed otherwise, you may use any built-in BSL
functions or data types.

\vskip 1em

\noindent
When writing tests, you may use a shorthand for writing check-expects
by drawing an arrow between two expressions to mean you expect the
first to evaluate to same result as the second.  For example, you may
write \verb|(add1 3)| $\rightarrow$ \verb|4| instead of
\verb|(check-expect (add1 3) 4)|.

\newpage

\noindent
{\bf Problem 1 (10 points).}
%
For the following program, write out each step of computation. At each
step, underline the expression being simplified. Label each step as
being ``arithmetic'' (meaning any built-in operation),
``conditional'', ``plug'' (for plugging in an argument for a function
parameter), or ``constant'' for replacing a constant with its value.


\begin{verbatim}
(define Q 1)
(define (w z) (< (string-length z) 20))
(w (cond [(= 1 Q) "fred"]
         [else 9]))
\end{verbatim}


\begin{SaveVerbatim}{VerbEnv}  

SOLUTION:

(w (cond [(= 1 Q) "fred"] [else 9])) -->[const]
               ^
(w (cond [(= 1 1) "fred"] [else 9])) -->[arith]
          ^^^^^^^
(w (cond [#true "fred"] [else 9]))   -->[cond]
   ^^^^^^^^^^^^^^^^^^^^^^^^^^^^^^
(w "fred")                           -->[plug]
^^^^^^^^^^
(< (string-length "fred") 20))       -->[arith]
   ^^^^^^^^^^^^^^^^^^^^^^
(< 5 20)                             -->[arith]
#true
\end{SaveVerbatim}

%\censor{\vspace{-2.5in}}
\censor{%
\noindent\BUseVerbatim{VerbEnv}}

\newpage

\noindent
{\bf Problem 2 (10 points).}
%
For the following structure definition, list the names of every
function it creates.  For each function, classify it as being either a
constructor, accessor, or predicate.

\begin{verbatim}
(define-struct dr (who strange))
\end{verbatim}

\begin{SaveVerbatim}{VerbEnv}  
SOLUTION:

- make-dr    : Constructor
- dr-who     : Accessor
- dr-strange : Accessor
- dr?        : Predicate
\end{SaveVerbatim}

%\censor{\vspace{-2.5in}}
\censor{%
\noindent\BUseVerbatim{VerbEnv}}

\newpage

\noindent
{\bf Problem 3 (15 points).}
%
You've been hired by the CMNS development office and been given their
existing software for spamming potential donors.  It contains the following
data definition:
\begin{verbatim}
;; A Person is a (make-name Title String String)
;; Interp: a person's title and first and last names.
(define-struct name (title first last))

;; A Title is one of:
;; - "r"      Interp: Mr.
;; - "s"              Ms.
;; - "d"              Dr.
\end{verbatim}

\noindent
Finish the design of the following function for creating letter
openings:
\begin{verbatim}
;; dear : Person -> String
;; Create a letter opening for the given person.
(check-expect (dear (make-name "d" "Minnie" "Maisy"))
              "Dear Dr. Maisy:")
(check-expect (dear (make-name "r" "Fred" "Rogers"))
              "Dear Mr. Rogers:")
(check-expect (dear (make-name "s" "Miriam" "Maisel"))
              "Dear Ms. Maisel:")
\end{verbatim}


\begin{SaveVerbatim}{VerbEnv}

SOLUTION:

(define (swap s i)
  (string-append (substring s i)
                 (substring s 0 i)))
\end{SaveVerbatim}

\censor{%
\noindent
\BUseVerbatim{VerbEnv}}

\newpage
\noindent
{\bf Problem 4 (15 points).}  
%
Design a program that takes a list of numbers and produces the sum of
positive numbers in the list.  For example, if the list contains 3, 4,
-1, 0, -2, and 5, the sum of positive numbers (3+4+5) is 12.  You may
use the following data definition:

\begin{verbatim}
;; A LoN (list of numbers) is one of:
;; - '()
;; - (cons Number LoN)
;; Interp: an arbitrarily long list of numbers
\end{verbatim}


\begin{SaveVerbatim}{VerbEnv}

SOLUTION:

;; sum-pos : LoN -> Number
;; Count the number of positive numbers in list
(check-expect (sum-pos '()) 0)
(check-expect (sum-pos (list 1 -2 8 0)) 9)
(define (sum-pos lon)
  (cond [(empty? lon) 0]
        [(cons? lon)
         (if (< (first lon) 0)
             (sum-pos (rest lon))
             (+ (first lon) (sum-pos (rest lon))))]))
\end{SaveVerbatim}

\censor{%
\noindent
\BUseVerbatim{VerbEnv}}



\newpage

\noindent
{\bf Problem 5 (40 points).} You've been hired by Nintendo to design
their next blockbuster game: Dots.  It consists of, well, dots: lots
and lots of dots, each of which occupies a coordinate in space.  You
come up with the following data definitions for representing a single
dot and a collection of dots:

\begin{verbatim}
;; A Dots is one of:
;; - empty
;; - (cons Dot Dots)
;; Interp: collection of an arbitrary number of dots

;; A Dot is a (make-posn Number Number)
;; Interp: a dot's position in space
\end{verbatim}

\vspace{1em}

\noindent
{\bf Problem 5(a) [10 points].} Write templates for \verb|Dot| and
\verb|Dots| functions.

\begin{SaveVerbatim}{VerbEnv}
  
SOLUTION:

(define (dot-template d)
  (... (posn-x d)
       (posn-y d)
       ...))

(define (dots-template ds)
  (cond [(empty? ds) ...]
        [(cons? ds) (... (dot-template (first ds))
                         (dots-template (rest ds)) ...)]))
\end{SaveVerbatim}

\vskip 18em

\censor{\vspace{-15em}}
\censor{%
\noindent\BUseVerbatim{VerbEnv}}


\noindent
{\bf Problem 5(b) [10 points].} Write a stub for the following function:
    
\begin{verbatim}
;; nw : Dots -> Dots
;; Move given dots in the northwest direction one unit
(check-expect (nw empty) empty)
(check-expect (nw (cons (make-posn 3 4) empty))
              (cons (make-posn 4 5) empty))
\end{verbatim}


\begin{SaveVerbatim}{VerbEnv}
  
SOLUTION:

(define (nw ds) empty)
        
\end{SaveVerbatim}

\censor{%
\noindent\BUseVerbatim{VerbEnv}}

\newpage

\noindent
{\bf Problem 5(c) [20 points].}
%
Design two functions: {\tt dot-flip} reflects a given dot over the
y-axis, for example, $(3,4)$ becomes $(-3,4)$, and {\tt dots-flip}
reflects a collection of dots over the y-axis.
\begin{SaveVerbatim}{VerbEnv}

;; SOLUTION:
  
;; dot-flip : Dot -> Dot
;; Flip dot over y-axis
(check-expect (dot-flip (make-posn 3 4)) (make-posn -3 4))
(define (dot-flip d)
  (make-posn (- (posn-x d)) (posn-y d)))
  
;; dots-flip : Dots -> Dots
;; Flip dots over y-axis
(check-expect (dots-flip empty) empty)
(check-expect (dots-flip (cons (make-posn 3 4) empty))
              (cons (make-posn -3 4) empty))
(define (dots-flip ds)
  (cond [(empty? ds) empty]
        [(cons? ds) 
         (cons (dot-flip (first ds))
               (dots-flip (rest ds)))]))                   
\end{SaveVerbatim}

\censor{%
\noindent
\BUseVerbatim{VerbEnv}}
  
\newpage

\noindent
[Extra space, should you need it.]


%% Design a program that consumes an arbitrarily large collection of
%% numbers and produces the count of non-negative numbers in the
%% collection.

%% \begin{SaveVerbatim}{VerbEnv}

%% SOLUTION:

%% ;; A LoN (list of numbers) is one of:
%% ;; - '()
%% ;; - (cons Number LoN)

%% ;; count-non-neg : LoN -> Natural
%% ;; Count the number of non-negative numbers in the given list.
%% (check-expect (count-non-neg '()) 0)
%% (check-expect (count-non-neg (cons -1 (cons 1 (cons -2 '())))) 1)
%% (define (count-non-neg lon)
%%   (cond [(empty? lon) 0]
%%         [(cons? lon) 
%%          (+ (if (<= 0 (first lon)) 1 0)
%%             (count-non-neg (rest lon)))]))
%% \end{SaveVerbatim}

%% \censor{%
%% \noindent
%% \BUseVerbatim{VerbEnv}}

\end{document}

