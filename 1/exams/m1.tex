\documentclass[12pt]{article}
\usepackage{fullpage}
\usepackage{times}
\usepackage{xcolor,censor}
\usepackage{fancyvrb}

\censorruleheight=0ex
%\StopCensoring

\begin{document}

\begin{center}
  {\Large {\bf CMSC 131A, Midterm 1}\\
    \censor{{\bf SOLUTION}} \ \\
    \ \\
    Fall 2017 \ \\
  }
\end{center}

\vspace{2em}

\noindent
NAME:\verb|________________________________________________________|

\vspace{2em}

\noindent
UID: \verb|__________________________________________|

\vspace{2em}

\begin{center}
\begin{tabular}{| c | c |}
  \hline
  Question & Points \\ \hline \hline
  1 & 10 \\ \hline
  2 & 10 \\ \hline
  3 & 10 \\ \hline
  4 & 15 \\ \hline
  5 & 15 \\ \hline
  6 & 10 \\ \hline
  7 & 20 \\ \hline \hline
  Total: & 90\\
  \hline
\end{tabular}
\end{center}

\noindent
This test is open-book, open-notes, but you may not use any computing
device other than your brain and may not communicate with anyone.
You have 50 minutes to complete the test.

\vskip 1em

\noindent
The phrase ``design a program'' or ``design a function'' means follow
the steps of the design recipe.  Unless specifically asked for, you do
not need to provide intermediate products like templates or stubs,
though they may be useful to help you construct correct solutions.

\vskip 1em

\noindent
You may use any of the data definitions given to you within this exam
and do not need to repeat their definitions.

\vskip 1em

\noindent
Unless specifically instructed otherwise, you may use any built-in BSL
functions or data types.

\vskip 1em

\noindent
When writing tests, you may use a shorthand for writing check-expects
by drawing an arrow between two expressions to mean you expect the
first to evaluate to same result as the second.  For example, you may
write \verb|(add1 3)| $\rightarrow$ \verb|4| instead of
\verb|(check-expect (add1 3) 4)|.

\newpage

\noindent
{\bf Problem 1 (10 points).}
%
For the following program, write out each step of computation. At each
step, underline the expression being simplified. Label each step as
being ``arithmetic'' (meaning any built-in operation),
``conditional'', ``plug'' (for plugging in an argument for a function
parameter), or ``constant'' for replacing a constant with its value.


\begin{verbatim}
(define R 9)
(define (w z) (= R z))
(cond [(w 2) 4]
      [else 9])
\end{verbatim}

\vskip 2.5in


\begin{SaveVerbatim}{VerbEnv}  
SOLUTION:
\end{SaveVerbatim}

%\censor{\vspace{-2.5in}}
\censor{%
\noindent\BUseVerbatim{VerbEnv}}

\newpage

\noindent
{\bf Problem 2 (10 points).}
%
For the following structure definition, list the names of every
function it creates.  For each function, classify it as being either a
constructor, accessor, or predicate.

\begin{verbatim}
(define-struct run (time up score))
\end{verbatim}

\newpage

\noindent
{\bf Problem 3 (10 points).}
%
The graphical text editor you developed in a recent assignment has
attracted the interest of investors.  But before they write you a
check, they want to make sure you can handle fancy operations for
expert text editor users.  They ask if you can implement a \emph{swap}
operation, which will swap content to the left of the cursor with the
content to the right..  To satisfy your potential investors, you
define the following function for swapping strings at a given
position:
\begin{verbatim}
;; swap : String Integer -> String
;; Swap string to the left of i with string to the right.
;; Assume (<= 0 i (string-length s)).
(check-expect (swap "" 0) "")
(check-expect (swap "ab" 1) "ba")
(check-expect (swap "bannana" 3) "nanaban")
(define (swap s i) ...)
\end{verbatim}

\noindent
Give a correct definition for \verb|swap|.  (You only need to
provide code, not the design steps.)

\begin{SaveVerbatim}{VerbEnv}

SOLUTION:

(define (swap s i)
  (string-append (substring s i)
                 (substring s 0 i)))
\end{SaveVerbatim}

\censor{%
\noindent
\BUseVerbatim{VerbEnv}}



\newpage 
\noindent
{\bf Problem 4 (15 points).}
%
Use the following data definition for representing names:
\begin{verbatim}
;; A Name is a (make-name String String)
;; Interp: a person's full (first and last) name.
;; Both strings must contain at least one letter.
(define-struct name (first last))
\end{verbatim}

\noindent
Write a program, \verb|name-width|, that determines the width needed
to print someone's name on a letter using a mono-typed font (meaning
each letter has the same width).  The program can calculate this width
based on the length of the person's first and last name, plus the
width of a space to go in between.  Assume each letter is 10 pixels
wide.  For example, the name Bob Harper requires 100 pixels; the name
Bo Jack requires 70 pixels.  You may start from the following header:

\begin{verbatim}
;; name-width : Name -> Number
;; Computes width needed to print name in mono-type font
\end{verbatim}

\begin{SaveVerbatim}{VerbEnv}

SOLUTION:
  
;; dist3d : 3D -> Number
;; Compute distance to origin from given 3-d point
(check-expect (dist3d (make-3d 3 2 6)) 7)
(define (dist3d p)
  (sqrt (+ (sqr (3d-x p)) (sqr (3d-y p)) (sqr (3d-z p)))))  
\end{SaveVerbatim}

\censor{%
\noindent
\BUseVerbatim{VerbEnv}}


\newpage
\noindent
{\bf Problem 5 (15 points).}  
%
Design a program that takes a list of numbers and produces the count
of non-negative numbers in the list.  For example, if the list
contains 3, 4, -1, 0, -2, and 5, the count of non-negative numbers is
4.  You may use the following data definition:

\begin{verbatim}
;; A LoN (list of numbers) is one of:
;; - '()
;; - (cons Number LoN)
;; Interp: an arbitrarily long list of numbers
\end{verbatim}





\begin{SaveVerbatim}{VerbEnv}

SOLUTION:
  
\end{SaveVerbatim}

\censor{%
\noindent
\BUseVerbatim{VerbEnv}}



\newpage

\noindent
{\bf Problem 6 (10 points).} You've been hired by the US Department of
Transportation to design traffic simulation software, which includes
simulating traffic lights.  Discovering the simple design from 131A is
insufficient, you come up with the following representation, which
makes it possible to model not just the color, but also how much
longer the current light has before changing: 

\begin{verbatim}
;; A Light is one of:
;; - (make-red Natural)
;; - (make-yellow Natural)
;; - (make-green Natural)
;; Interp: color and remaining duration (in ticks) of a light 
(define-struct red (count))
(define-struct yellow (count))
(define-struct green (count))
\end{verbatim}
(Recall that a \verb|Natural| is a natural number, i.e. a non-negative
integer.)

\vspace{1em}

\noindent
{\bf Problem 6(a).} Write a template for \verb|Light| functions.

\begin{SaveVerbatim}{VerbEnv}
  
SOLUTION:
  
(define (light-temp l)
  (cond [(red? l) (... (red-count l) ...)]
        [(yellow? l) (... (yellow-count l) ...)]
        [(green? l) (... (green-count l) ...)]
        
\end{SaveVerbatim}

\vskip 15em

\censor{\vspace{-15em}}
\censor{%
\noindent\BUseVerbatim{VerbEnv}}


\noindent
{\bf Problem 6(b).} Write a stub for the following function:
    
\begin{verbatim}
;; tock : Light -> Light
;; Advance the light by one tick of time
(check-expect (tock (make-red 10)) (make-red 9))
(check-expect (tock (make-red 0)) (make-green 10))
\end{verbatim}


\begin{SaveVerbatim}{VerbEnv}
  
SOLUTION:

(define (tock l) (make-red 0))
        
\end{SaveVerbatim}

\censor{%
\noindent\BUseVerbatim{VerbEnv}}

\newpage

\noindent
{\bf Problem 7 (20 points).}
%
Over the summer, you take an internship at the University of
Maryland's Center for Bioinformatics and Computational Biology (CBCB).
Your first task deals with computing information about \emph{DNA
  strands}.  A DNA strand is an arbitrarily long sequence of
\emph{nucleotides}, which are either: C, G, A, or T.  Design a program
that, given a DNA strand, computes the number of times the G
nucleotide occurs in it.  For example, if the strand is AGCTTTGA, your
program should produce 2.

\begin{SaveVerbatim}{VerbEnv}

;; SOLUTION:
  
;; A Nucleotide is one of:
;; - "C"
;; - "G"
;; - "A"
;; - "T"
;; Interp: DNA nucleotide C,G,A,T.

;; A DNA is one of:
;; - '()
;; - (cons Nucleotide DNA)
;; Interp: a DNA strand of nucleotides

;; count-g : DNA -> Natural
;; Count the number of occurrence of G in given DNA strand
(check-expect (count-g '()) 0)
(check-expect (count-g (cons "A" (cons "G" (cons "G" '())))) 2)
(define (count-g dna)
  (cond [(empty? dna) 0]
        [(cons? dna)
         (+ (if (string=? (first dna) "G") 1 0)
            (count-g (rest dna)))]))

\end{SaveVerbatim}

\censor{%
\noindent
\BUseVerbatim{VerbEnv}}
  
\newpage

\noindent
[Extra space, should you need it.]


%% Design a program that consumes an arbitrarily large collection of
%% numbers and produces the count of non-negative numbers in the
%% collection.

%% \begin{SaveVerbatim}{VerbEnv}

%% SOLUTION:

%% ;; A LoN (list of numbers) is one of:
%% ;; - '()
%% ;; - (cons Number LoN)

%% ;; count-non-neg : LoN -> Natural
%% ;; Count the number of non-negative numbers in the given list.
%% (check-expect (count-non-neg '()) 0)
%% (check-expect (count-non-neg (cons -1 (cons 1 (cons -2 '())))) 1)
%% (define (count-non-neg lon)
%%   (cond [(empty? lon) 0]
%%         [(cons? lon) 
%%          (+ (if (<= 0 (first lon)) 1 0)
%%             (count-non-neg (rest lon)))]))
%% \end{SaveVerbatim}

%% \censor{%
%% \noindent
%% \BUseVerbatim{VerbEnv}}

\end{document}

