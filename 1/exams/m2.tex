\documentclass[12pt]{article}
\usepackage{fullpage}
\usepackage{xcolor,censor}
\usepackage{fancyvrb}
\usepackage{fontspec}
\usepackage{alltt}

\setmainfont{Times New Roman}

\censorruleheight=0ex
%\StopCensoring

\begin{document}

\begin{center}
  {\Large {\bf CMSC 131A, Midterm 2}\\
    \censor{{\bf SOLUTION}} \ \\
    \ \\
    Fall 2017 \ \\
  }
\end{center}

\vspace{2em}

\noindent
NAME:\verb|________________________________________________________|

\vspace{2em}

\noindent
UID: \verb|__________________________________________|

\vspace{2em}

\begin{center}
\begin{tabular}{| c | c |}
  \hline
  Question & Points \\ \hline \hline
  1 & 15 \\ \hline
  2 & 12 \\ \hline
  3 & 20 \\ \hline
  4 & 15 \\ \hline
  5 & 15 \\ \hline
  6 & 20 \\ \hline\hline
  Total: & 97\\
  \hline
\end{tabular}
\end{center}

\noindent
This test is open-book, open-notes, but you may not use any computing
device other than your brain and may not communicate with anyone.
You have 50 minutes to complete the test.

\vskip 1em

\noindent
The phrase ``design a program'' or ``design a function'' means follow
the steps of the design recipe.  Unless specifically asked for, you do
not need to provide intermediate products like templates or stubs,
though they may be useful to help you construct correct solutions.

\vskip 1em

\noindent
You may use any of the data definitions given to you within this exam
and do not need to repeat their definitions.

\vskip 1em

\noindent
Unless specifically instructed otherwise, you may use any built-in ISL+
functions or data types.

\vskip 1em

\noindent
When writing tests, you may use a shorthand for writing check-expects
by drawing an arrow between two expressions to mean you expect the
first to evaluate to same result as the second.  For example, you may
write \verb|(add1 3)| $\rightarrow$ \verb|4| instead of
\verb|(check-expect (add1 3) 4)|.

\newpage

\noindent
{\bf Problem 1 (15 points).}
%
Here are some data definition relevant to representing a dictionary,
which associates words with the their definitions:

\begin{verbatim}
;; A Dict is one of:
;; - '()
;; - (cons (list String String) Dict)
;; Interp: a collection of definitions where each element is a 
;; two-element list of a word (first) and its meaning (second).

;; A Result is one of:
;; - #false
;; - String
;; Interp: represents the result of looking up a word in the
;; dictionary; #false means not found.
\end{verbatim}

\noindent
Design the following function:

\begin{verbatim}
;; lookup : String Dict -> Result
;; Lookup the (first) definition of a word in the dictionary,
;; produces #false if not found
\end{verbatim}

\begin{SaveVerbatim}{VerbEnv}

SOLUTION:

(define (f b s) s)
(define (g fs) (fs #false))
(define (h bs ns) 2)
(define (i f) (f 2))
(define (j f xs) (map f xs))
(define (k f g x) (g (f x)))
\end{SaveVerbatim}

%\censor{\vspace{-2.5in}}
\censor{%
\noindent\BUseVerbatim{VerbEnv}}

\newpage

\noindent
{\bf Problem 2 (12 points).}
%
For each of the following functions, provide the most general signature
that correctly describes the function:

\begin{alltt}
;; method : \censor{Number -> Number}

(define (method x)
  (+ (sqr x) 2))

;; raekwon : \censor{[Listof Number] -> Boolean}

(define (raekwon x)
  (ormap positive? x))

;; rza : \censor{[Listof String] [Listof Number] -> [Listof Number]}

(define (rza x y)
  (cond [(empty? x) y]
        [(cons? x)
         (cons (string-length (first x))
               (rza (rest x) y))]))

;; gza : \censor{(String Number -> Number) -> Number}

(define (gza x)
  (foldr x 0 (list "a" "b" "c")))

;; odb : \censor{(Number -> Number) -> (Number -> Number)}

(define (odb x)
  (lambda (y)   
    (/ (- (x (+ y 0.001))
          (x (- y 0.001)))
       (* 2 0.001))))
    

;; ghostface : \censor{[X Y] [Listof X] [X -> Y] -> [Listof Y]}

(define (ghostface x y)
  (cond [(empty? x) '()]
        [(cons? x)
         (cons (y (first x)) 
               (ghostface (rest x) y))]))
\end{alltt}


\newpage

\noindent
{\bf Problem 3 (20 points).}
%
Design a program called {\tt w-avg} that computes the weighted average
of a list of numbers and a list of weights.

For example, let's say a class grade is based on two midterms and a
project where the project is worth twice as much as the midterms.  A
student who gets an 80 and a 70 on the midterms and a 90 on the
project would have a weighted average of 82.5: $((1 \times 80) + (1
\times 70) + (2 \times 90)) / 4$, which can be computed with {\tt
  (w-avg (list 80 70 90) (list 1 1 2))}.  You should assume the two
lists have the same length.


\begin{SaveVerbatim}{VerbEnv}

SOLUTION:

;; avg : [Listof Number] -> Number
;; Compute the average of a list of numbers
;; Assume: the list is non-empty
(check-expect (avg (list 3 4 5)) 4)
(check-expect (avg (list 10 20)) 15)
(define (avg lon)
  (/ (sum lon) 
     (length lon)))

;; sum : [Listof Number] -> Number
;; Sum the list of numbers
(check-expect (sum '()) 0)
(check-expect (sum (list 3 4 5)) 12)
(define (sum lon)
  (foldr + 0 lon))

;; Alt: follow [Listof Number] template
\end{SaveVerbatim}

\censor{%
\noindent
\BUseVerbatim{VerbEnv}}


\newpage 
\noindent
{\bf Problem 4 (15 points).}
%
Here is a parametric data definition for a tree of elements:
\begin{verbatim}
;; A [Tree X] is one of:
;; - (make-leaf)
;; - (make-node X [Tree X] [Tree X])
;; Interp: a binary tree that is either empty (a leaf), or non-empty (a node)
;; with an element and two sub-trees.
(define-struct leaf ())
(define-struct node (elem left right))
\end{verbatim}
%% {\bf Problem 4(a) (10 points).} Design a function, {\tt tree-sum},
%% that operates on a binary tree of numbers, {\tt [Tree Number]}, and
%% produces the sum of all the numbers in the tree.

%% \begin{SaveVerbatim}{VerbEnv}


%% SOLUTION:

%% ;; tree-sum : [Tree Number] -> Number
%% ;; Sum all the numbers in the tree
%% (check-expect (tree-sum (make-leaf)) 0)
%% (check-expect (tree-sum (make-node 7 (make-node 1 (make-leaf) (make-leaf)) 
%%                                      (make-leaf)))
%%               8)
%% (define (tree-sum bt)
%%   (cond [(leaf? bt) 0]
%%         [(node? bt)
%%          (+ (node-elem bt)
%%             (tree-sum (node-left bt))
%%             (tree-sum (node-right bt)))]))

%% \end{SaveVerbatim}

%% \censor{%
%% \noindent
%% \BUseVerbatim{VerbEnv}}


%% \newpage

\noindent
%{\bf Problem 4(b) (10 points).}
Here is a function that produces a list of all the elements in a tree:
\begin{verbatim}
;; tree-elems : [X] . [Tree X] -> [Listof X]
;; Produce a list of all elements in tree going top-down, left-to-right
(check-expect (tree-elems (make-leaf)) '())
(check-expect (tree-elems (make-node 7 
                                     (make-node 9 (make-leaf) (make-leaf))
                                     (make-node 2 (make-leaf) (make-leaf))))
              (list 7 9 2))
(define (tree-elems bt)
  (cond [(leaf? bt) '()]
        [(node? bt)
         (cons (node-elem bt)
               (append (tree-elems (node-left bt))
                       (tree-elems (node-right bt))))]))
\end{verbatim}
Here is an abstraction function for trees that is similar to
{\tt foldr} for lists, but works on trees:
\begin{verbatim}
;; tree-fold : [X Y] . [X Y Y -> Y] Y [Tree X] -> Y
;; The fundamental abstraction function for trees
(define (tree-fold f b bt)
  (cond [(leaf? bt) b]
        [(node? bt)
         (f (node-elem bt)
            (tree-fold f b (node-left bt))
            (tree-fold f b (node-right bt)))]))
\end{verbatim}
Give an equivalent definition of {\tt tree-elems} in terms of {\tt
  tree-fold}.  (Just provide the code.)

\vspace{1em}
\noindent
[Provide your answer on the next page.]

\begin{SaveVerbatim}{VerbEnv}


SOLUTION:

(define (tree-elems bt)
  (tree-fold (lambda (x l r) (cons x (append l r))) '() bt))

\end{SaveVerbatim}

\censor{%
\noindent
\BUseVerbatim{VerbEnv}}

\newpage

\noindent
[Space for problem 4.]


\newpage
\noindent
{\bf Problem 5 (15 points).}  
%
Design a program that takes a list of strings and produces the count
of strings with length longer than 3 in the list.  For example, if the
list contains {\tt "a"}, {\tt "dave"}, {\tt "abc"}, and {\tt "fred"},
the count is 2.  You may assume the {\tt{[Listof String]}} data
definition is defined.
%
For full credit, use list abstraction functions.  For partial credit,
follow the template for {\tt{[Listof String]}}.


\begin{SaveVerbatim}{VerbEnv}


SOLUTION:

;; count-3+ : [Listof String] -> Number
;; Count the number of strings with length more than 3 in given list
(check-expect (count-3+ '()) 0)
(check-expect (count-3+ (list "a" "abcd" "fred" "c")) 2)
(define (count-3+ los)
  (foldr (lambda (s c) (if (> (string-length s) 3) (add1 c) c)) 0 los))

;; Alts:
(define (count-3+ los)
  (local [;; count+ : String Number -> Number
          ;; Bump count if string has length > 3
          (define (count+ s c) 
            (if (> (string-length s) 3) (add1 c) c))]
    (foldr count+ 0 los))

(define (count-3+ los)
  (local [;; 3+? : String -> Boolean
          ;; Does the string have length > 3?
          (define (3+? s) 
            (> (string-length s) 3))]
    (length (filter 3+? los))))

(define (count-3+ los)
  (length (filter (lambda (s) (> (string-length s) 3)) los)))
\end{SaveVerbatim}

\censor{%
\noindent
\BUseVerbatim{VerbEnv}}




\newpage

\noindent
{\bf Problem 6 (20 points).}
%
Here's a data definition for choose-your-own-adventure books:
\begin{verbatim}
;; A CYOA is one:
;; - String
;; - (make-choice String CYOA CYOA)
(define-struct choice (q yes no))
;; Interp: a choose-your-own-adventure book where a string
;; is a conclusion and a choice is a yes/no question and
;; the two adventures that follow as a consequnce of the answer.
\end{verbatim}

\noindent
Design a function called {\tt ans} that consumes a {\tt CYOA} and a
{\tt [Listof Boolean]} that represents answers to the yes/no
questions: {\tt \#true} means yes, {\tt \#false} means no.  The {\tt
  ans} function produces the {\tt CYOA} adventure that results after
giving all of the answers in the list.  If the list contains too many
answers, {\tt ans} just produces the conclusion of the adventure.


\begin{SaveVerbatim}{VerbEnv}

SOLUTION:

(define a0 "A0")
(define a1 (make-choice "Q1" "A1" "A2"))
(define a2 (make-choice "Q2" a0 a1))

;; ans : CYOA [Listof Boolean] -> CYOA
;; ...
(check-expect (adv a0 (list #true #false)) "A0")
(check-expect (adv a1 (list #true #false)) "A1")
(check-expect (adv a2 (list #false #true)) "A1")
(check-expect (adv a2 (list #false)) a1)
(define (ans adv loa)
  (cond [(string? adv) adv]
        [(empty? loa) adv]
        [(first loa)
         (ans (choice-yes adv) (rest adv))]
        [else
         (ans (choice-no adv) (rest adv))]))
\end{SaveVerbatim}

\censor{%
\noindent
\BUseVerbatim{VerbEnv}}


\end{document}

